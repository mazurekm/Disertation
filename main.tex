% Szkielet dla pracy pisanej w języku angielskim.

\documentclass[english,a4paper,oneside]{ppfcmthesis}

\usepackage{amssymb}
\usepackage[utf8]{inputenc}
\usepackage[OT4]{fontenc}
\usepackage{cite}
\usepackage{indentfirst}
\usepackage{listings}
\usepackage{pgfplots}
\usepackage{pgfplotstable}
\usepackage{tikz}
\usepackage{graphicx}
\usepackage{subcaption}
\usepackage{appendix}
\usepackage{float}

\lstset{language=C++}

\authortitle{}
\author{Michał Mazurek}                              
\title{Efficient and scalable algorithms for multi-label classification based on feature and label space transformation}           
\ppsupervisor{~dr~inż.~Krzysztof Dembczyński} 
\ppyear{2016}                                  

\begin{document}

% Front matter starts here
\frontmatter\pagestyle{empty}%
\maketitle\cleardoublepage%

% Blank info page for "karta dyplomowa"
\thispagestyle{empty}\vspace*{\fill}%
\begin{center}Tutaj przychodzi karta pracy dyplomowej;\\oryginał wstawiamy do wersji dla archiwum PP, w pozostałych kopiach wstawiamy ksero.\end{center}%
\vfill\cleardoublepage%

\pagenumbering{Roman}\pagestyle{ppfcmthesis}%

%Abstract
\addcontentsline{toc}{chapter}{Abstract}

\begin{abstract}
\thispagestyle{plain}
    Your abstract goes here...
    ...
\end{abstract}
\cleardoublepage


%Acknowledgements
\addcontentsline{toc}{chapter}{Acknowledgements}

\chapter*{\centerline{Acknowledgement}}
\thispagestyle{plain}
   
I would first like to thank my thesis advisor Ph.D. Krzysztof Dembczyński of the Laboratory of Intelligent Decision Support Systems at Poznan University of Technology. The door to Prof. Dembczyński office was always open whenever I ran into a trouble spot or had a question about my research or writing. He consistently allowed this paper to be my own work, but steered me in the right the direction whenever he thought I needed it.

I must also express my very profound gratitude to my parents for providing me with unfailing support and continuous encouragement throughout my years of study and through the process of researching and writing this thesis. This accomplishment would not have been possible without them. Thank you.

\cleardoublepage


% Table of contents.
\tableofcontents* \cleardoublepage%

\cleardoublepage
\listoffigures

\cleardoublepage
\listoftables


% Main content of your thesis starts here.
\mainmatter%

\chapter{Introduction}

\section{Motivation}

In 1855 Mathew Fontaine Maury, nicknamed `Pathfinder of the Seas', published his impressive dissertation titled \textit{The Physical Geography of the Sea}. The thesis begun the revolution in a sea transport system which allowed to shorten time of long travels and to save a lot of resources. Before publishing the book, Maury collected and processed almost 1.2 million of single data. It was certainly very arduous work, because people had to make the calculation instead of computers. The data came mainly from journey log books written by captains of United States Navy. The `Pathfinder of the Seas' was one of the firsts who realized that there is a special value hidden in huge sets of data. It just needs to be explored \citep{Data}.  

160 years after Maury's discoveries, we notice the meaning of `big data' almost everywhere: starting from pure science passing through finance, medicine or industry ending up at social media. Computers are capable of predicting whether a patient, who has had heart attack, will have a second heart attack or identify factors which are responsible for prostate cancer. It is also possible to predict a price of stock in 6 months from now, on the basis of company performance and economic data etc.
It is worth emphasizing that such deep analyses are achieved through fast computing systems. Nowadays processing of 1.2 million records, which had been being analyzed by Maury and his collaborators for probably a few years, takes just a while. Collecting required information is not a problem as well. According to information from \textit{IBM} website,\footnote{\bibentry{IBM}} every day we create 2.5 quintillion bytes of data. The data comes from various sources: sensors used to gather climate information, posts to social media sites, digital pictures and videos, cell phones \textit{GPS} signal, and so on.

As we see, the only thing left to do is to make sense of collected data: to extract important patterns and trends, and understand `what the data says'. In other words, we need to learn from data. The branch of science which studies the problem of `learning from data' is called machine learning. This thesis is devoted to algorithms which solve the multi-label classification task which is one of the fundamental problems of machine learning. The details connected with machine learning aspects, especially with the multi-label classification, are presented in the next section.


\section{Machine learning tasks}

Machine learning is a branch of computer science which focuses on creating an automatic system which can learn and improve itself from a gathered experience (data). There are three main types of machine learning tasks:

\begin{itemize}
\item supervised learning,
\item unsupervised learning,
\item reinforcement learning.
\end{itemize}

In the supervised learning the experience is represented by a set of exemplary input data with desired outputs. This set is provided by a `teacher' and the main goal is to find rules which allow predicting an output from a given input.

The reinforcement learning is a more complicated issue. In this case we usually consider a dynamical environment with various states and an agent which can perform one of available actions in order to achieve a goal. In contrast to the supervised learning, there is no direct information about correct input-output pairs. As a result, an environment must be explored by an agent in order to gain a knowledge. In other words instead of a `teacher' there is a `nitpicker' which awards an agent for making a particular action. 

The last type of machine learning tasks, the unsupervised learning, aims at finding hidden structures in unlabelled data. In this case there are no error and reward signals which could be used to evaluate a solution. In fact, this is the most difficult and also the most challenging aspect of machine learning. 

In this study, as it has been already mentioned, we consider multi-label classification which is one of the fundamental supervised learning problems. 

The general classification problem aims at matching a specific category (also called class) to a new observation, on the basis of a training set. A training set is a set of observations whose class, they belong to, is known. Let us notice that in the classification problem we can distinguish two separated problems: \textit{the binary classification} and \textit{the multi-class classification}. The difference between them is certainly connected with a number of classes. In the first case we have only two categories, whereas in the second one there are many of them. In other words, multi-class classification is an extension of binary classification for more than two categories. However, in both cases there is always one target (only one class is assigned to an object). If we increase a number of targets instead of a number of classes in binary classification, then we get the multi-label classification problem. In this case a subset of labels is predicted for a specific instance. Let us notice that there is also a relation between multi-class classification and multi-label classification. We can treat each subset of labels as an independent category and then simply transform the latter problem into the former one. 

It is worth emphasizing that the dependencies between particular classification problems are used to design most of multi-label algorithms. 

\section{Scope of thesis}

This thesis is devoted to multi-label methods which are strongly based on linear algebra. In practice this means that the algorithms use linear transformation to project a training set to a new reduced linear space at first, and then train linear regression in the reduced space. In the next chapters we can find mathematical background of these approaches, details of implementation and results of experiments performed on various data sets. However, we should consider if such an approach really makes sense at first. First of all let us walk through the most known concepts which are usually used to solve multi-label problems.

The simplest methods are based on transformation to multi-class problem. If we obtain a problem in such a form, we can solve it by one of standard learning algorithms (\textit{Naive Bayes classifier}, \textit{k-NN}, etc.). Unfortunately, such an approach has a significant defect which usually makes it useless. In fact, a number of classes grows exponentially with a number of labels!

The most popular approach is the \textit{binary relevance method (BR)} which transforms any multi-label problem into one binary problem for each label (it is sometimes also called \textit{one-against-all}). In fact, this method trains $k$ classifiers $C_1, \cdots, C_{k}$ -- each classifier $C_j$ predicts a corresponding label $l_j$.  The main problem in this approach is connected with ignoring hidden information in a space of labels. This hidden information involves the interdependencies between particular labels. On the other hand, BR is simple and has low computational complexity. Moreover, it can be effectively parallelized because we build classifiers independently for each label. In fact, these are its main strengths \citep{Chain}.

The algorithms presented in this dissertation extend the BR approach by preliminary projecting a training set to a new space. Let us notice that we can transform feature and label spaces into spaces of uncorrelated variables and then train a linear regressor for each label. In fact, we take into consideration this hidden information which is ignored by classic BR. Linear transformations can be also used to compress a training set in a sensible way. In fact, we save memory and time at the expense of losing accuracy. 

BR is certainly a very general approach, so using linear regression is not required. However, this algorithm is very fast if it is well implemented. Its high efficiency is connected with linear algebra operations such as matrix multiplication and inversion that can be effectively performed in a multi-threaded environment. As a result, even for huge training sets, a sensible training time might be achieved. 

It is worth emphasizing that for the purposes of the experiments the algorithms were implemented in C++11 programming language. This decision was dictated by an efficiency profit of using low-level programming language and a full control over a compilation process. Moreover, additional software tools have been implemented in order to perform computational experiments, for example, a parser for importing data sets in popular data formats (i.e. \textit{ARFF}) and a tool for measuring the predictive performance of the algorithms. As a result, these all elements were collected together and became a part of the library which can be used for multi-label classification.


\chapter{Background}
\section{Notation}

Vectors are denoted by lowercase letters $w$, $x$ etc. while matrices by uppercase letters $W$, $X$ etc. Moreover $x^{(i)}$ is an $i-th$ row of matrix, $(x^T)^{(j)}$ is a $j-th$ column of matrix $X$ and $x_{ij}$ is an element in $i-th$ row and $j-th$ column. $X \in \mathbb{R}^{m \times n}$ symbol means that matrix $X$ has $m$ rows and $n$ columns. For $c \times d$  matrix $X \in \mathbb{R}^{c \times d}$, $\norm{X}_{F}$ is used for its Frobenius norm, $X^P$ is its pseudoinverse, $A^T$ is matrix transposed to $A$. 
ehere are also several functions used in this paper, ecspecially in pseudocode sections:
\begin{itemize}
\item \textit{tr(X)} is a trace of matrix $X$, 
\item \textit{randn(n,k)} generates $n \times k$ random matrix using normal distribution,
\item \textit{eig(X)} returns eigenvalues and eigenvectors of matrix $X$,
\item \textit{round(X)} returns $0$ if $x$ is closer to $0$ than $1$ else it returns $1$,
\item \textit{orthogonalize(X)} returns orthonormal basis of range space of matrix $X$,
\item \textit{inv(X)} inverses matrix $X$
\item \textit{head\_cols(X,h)} returns first $h$ left columns of matrix $X$,
\item \textit{push\_front\_column($w^T$,X)} inserts column $w^T$ into $X$ as first column.

\end{itemize}

\section{Formal problem definition}
Mutli-label classification is the problem of finding the following function: 

\begin{equation}\label{eq:def}
    f: [c_1, c_2, c_3, \cdots] \rightarrow [l_1, l_2, l_3, \cdots], \text{ } c_i \in \mathbb{R} \text{ } l_i \in \{0,1\} 
\end{equation}

As we see this classifier maps the vector of features into the vector of labels in contrast to standard classification, where a scalar is predicted. In order to face the problem and find desired function, we can use one of two main approaches: problem transformation methods and algorithm adaptation methods. In the first approach, a multi-label issue is considered as a set of binary classification problems - for each label a single-class classifier is trained. The second method focuses on adapting existing solutions for the standard classification to the multi-label. In other words, the problem is not divided into the simpler binary problems - it is seen as whole.

In this paper, the presented algorithms belong to the problem transformation methods approach. The base solution uses linear regression to predict specific labels. The rest of methods make an enhancement which rely on transformations of the feature and the label spaces. 

\section{Solution based on linear regression model}

In this section we focus on solution based on linear regression model. Regression analysis aims at studying, speaking generally, dependency between variables. 

Let $X \in \mathbb{R}^{m \times  ( n+1 )}$ be design matrix, $Y \in \mathbb{R}^{m \times k}$ label matrix, $\beta \in \mathbb{R}^{(n+1) \times k}$ matrix of coefficents of the regressor and $\epsilon \in \mathbb{R}^{m \times k}$ matrix of errors. The relation between these variables is described by formula:

\begin{equation}\label{eq:LR1}
    Y = XW + \epsilon 
\end{equation}
In order to estimate $\beta$ matrix, you might minimize sum of squared residuals (SSR), what can be written as a following formula:

\begin{equation}\label{eq:LR2}
    \hat{\beta} = \argmin_{\beta \in \mathbb{R}^{(n+1) \times k}}\sum\limits_{i=1}^{m}\sum\limits_{j=1}^{k}(y_{ij}-x^{(i)}(w^T)^{(j)})^2
\end{equation}
The solution of above optimization problem is equivalent to \cite{Weisberg}:

\begin{equation}\label{eq:LR3}
    \hat{\beta} = (X^TX)^{P}X^TY
\end{equation}
Although $X^TX$ matrix is squared, there is no certainty that it is ivertable. Therefore pseudoinverse must be used. There are of course prons and cons of this approach. First of all this model is simple and rather easy to implement - this is its main strength. Apart from that, it poses a base for different methods (some of them are described in this paper). On the other hand, there are strong assumptions which must be taken into account during costructing the model:
\begin{itemize}
    \item we assume linearity between feature variables and particular labels,
    \item feature values are fixed - they are not random,
    \item the error coefficents in matrix error are uncorrelated with each other - they have normal distribution,
    \item the number of examples should be higher than the number of features in the design matrix,
    \item SSR values are not correlated.
\end{itemize}

From the machine learning point of view, the main problem of linear regression is posed by a tendency to overfitting. However we can deal with that by regularization (Tikhonov regularization in this case). Then the problem might be described by a following formula:

\begin{equation}\label{eq:LR2}
    \hat{\beta} = \argmin_{\beta \in \mathbb{R}^{(n+1) \times k}}\left[\sum\limits_{i=1}^{m}\sum\limits_{j=1}^{k}(y_{ij}-x^{(i)}(w^T)^{(j)})^2 + \lambda\sum\limits_{i=1}^{n}\sum\limits_{j=1}^{k}w_{ij}^2\right]
\end{equation}
where $\lambda$ is a regularization parameter, which is a control on fitting parameters.
The solution, taking into account the regularization is then equivalent to \cite{Tikh}:

\begin{equation}\label{eq:LR3}
    \hat{\beta} = (X^TX+\lambda
        \begin{bmatrix}
        0 & 0 & \cdots & 0 \\
        0 & 1 & \cdots & 0 \\
        \vdots & \vdots & \ddots & \vdots \\
        0 & 0 & 0 & 1
        \end{bmatrix}
    )^{-1}X^TY
\end{equation}
where the matrix following $\lambda$ is an $(n+1)\times(n+1)$ diagonal matrix.  

Both approaches certainly have similar complexity. There are two matrix operations used to create regressor: multiplying and inverting, which can be computed in $O(n^{2.373})$ time by algorithms based on Coppersmith–Winograd method \cite{VVW}. Let us note that, matrix $X^TX \in \mathbb{R}^{(n+1) \times (n+1)}$ is inverted, so reduction of feature space (less columns) decreases time complexity. It is also worth to consider the reduction of $Y$ - for many labels it affects time-consumption too.   

The full learning algorithm based on linear regression is as shown in Algorithm \ref{alg:LR} 

\begin{algorithm}
    \caption{Linear regression based classifier}\label{alg:LR}
    \begin{algorithmic}[1]
    \State $\text{Let } Z \text{ be a copy of input feature matrix } X$
    \State $Z=push\_front\_column(ones, Z) \text{ where } ones=[1, 1, \cdots]^T$
    \State Let $I \in \mathbb{R}^{(n+1) \times (n+1)}$ be an identity matrix
    \State $I[0][0] = 0$
    \State $W=inv(Z^TZ+\lambda I)Z^TY$
    \State Predict label-set of an instance $x$ by $h(x)=round(xW)$ 
    \end{algorithmic}
\end{algorithm}


\section{Feature and label space reduction with PCA}

\subsection{Standard PCA algorithm}

One of possible methods which can be used to reduce dimension of data set is Principal Component Analysis. This reduction can be achieved by transforming to a new set of variables, the principal components, which are uncorrelated, and which are ordered so that the first few retain most of variation present in all of the original variables. Computation of the principal components reduces to the solution of an eigenvalue-eigenvector for a positive-semifdefinite symmetric matrix (covariance or correlation matrix) \cite{Jolliffe}.

Finding PCs can be obtained by \textit{Singular Value Decomposition}. Let $X \in \mathbb{R}_{m \times n}$ be an input matrix where $n$ is a number of variables. Moreover let us assume that $X$ is centered (a mean in each column is equal to $0$). Covariance matrix $C \in \mathbb{R}_{n \times n}$ is then equal to $\frac{X^TX}{n-1}$. Taking into account that $C$ is a symmetric matrix, its SVD decomposition is equivalent to:

\begin{equation}\label{eq:pca1}
    C=V\Sigma{V^T}
\end{equation}
where $V \in \mathbb{R}^{m \times m}$ is an orthogonal matrix of eigenvectors (called also principal axes), while $\Sigma \in \mathbb{R}^{m \times m}$ is diagonal matrix with eigenvalues in the decreasing order on its diagonal. 
Let us now consider SVD decomposition of $X$:

\begin{equation}\label{eq:pca2}
    X=V^{'}\Sigma^{'}{V^{'T}}
\end{equation}
It is easy to see that covariance matrix $C$ might be expressed by SVD decomposition of $X$:

\begin{equation}\label{eq:pca3}
    C=\frac{X^TX}{n-1} = \frac{(U^{\prime}\Sigma^{\prime}V^{\prime T})^T(U^{\prime}\Sigma^{\prime}V^{\prime T})}{n-1} = \frac{V^{\prime}\Sigma^{\prime}U^{\prime T}U^{\prime}\Sigma^{\prime}V^{\prime T}}{n-1} = V^{\prime}\frac{\Sigma^{\prime 2}}{n-1}V^{\prime T} 
\end{equation}
In fact, there is no need to use covariance matrix to compute its principals axes and its eigenvalues - SVD of $X$ matrix is enough \cite{Jolliffe}. In order to obtain new space for data (Prinicipal Components), the following equation can be used:

\begin{equation}\label{eq:pca4}
    PCs = XV = (U^{'}\Sigma^{'}V^{'T})V = U^{'}\Sigma^{'} 
\end{equation}
The reduction of data can be achieved by leaving first $h$ columns of $PCs$ matrix, where $h<n+1$.

PCA features and/or labels reduction can be simply combined with classifier based on linear regrssion model. At first a data are transformed into principal components to be reduced and then a regressor can be learnt from such encoded data. If a classifier is learnt from encoded label space, then a score of prediction must be decoded to an appropriate space. All required steps are presented in Algorithm \ref{alg:pca1}.

\begin{algorithm}
    \caption{Multi-dimension linear regressor with preliminary data reduction}\label{alg:pca1}
    \begin{algorithmic}[1]
    \Function{computePCA}{$X \in \mathbb{R}^{m \times n}$} 
        \State $\text{Let } Z=[z_1 \cdots z_m] \text{ with } z_i = x_i-\bar{x}$
        \State $\text{Perform svd on } Z \text{ to obtain } Z = U\Sigma{V^T}$
        \State $\text{return } (U,\Sigma,V)$
    \EndFunction
    \item[] 
    \State Let $X^{*}$ be a copy of $X$ and $Y^{*}$ be a copy of $Y$
    \If{reduceX == true}
        \State $(U_x, \Sigma_{x}, V_x) = computePCA(X,h1)$
        \State $X^{*}=head\_cols(X, h1)$
    \EndIf
    \If{reduceY == true}
        \State $(U_y, \Sigma_{y}, V_y) = computePCA(Y,h2)$
        \State $Y^{*}=head\_cols(Y, h2)$
    \EndIf
    \State Learn linear regressor $r(X^*)$ from $\{X^*,Y^*\}$
    \item[]
    \State Let $x^*$ be a copy of instance $x$
    \If{reduceX == true}
        \State $x^* = x^*V_x$
    \EndIf
    \If{reduceY == true}
        \State Predict the label-set of $x^*$ by $h(x^*)=round( r(x^*) * head\_rows(V_y^T, h2) + \bar{y})$ 
    \Else
        \State Predict the label-set of $x^*$ by $h(x^*)=round( r(x^*) )$ 
    \EndIf

    \end{algorithmic}
\end{algorithm}

Creating a regressor from reduced data has certainly less complexity than from original data, however PCA algorithm costs a time. The implementation based on SVD decomposition for $m\times{n}$ matrix is $O(m^2n + n^3)$, so it is hard to expect this method will work faster. It is also important to remember, that PCA might be understood as a lossy compression. For some data sets it might prevent an algorithm from being overfitted, but it might also decrease its quality - it is naturally connected to compression degree. 
One of idea to deal with complexity issue is to use faster PCA algorithm, which is not so accurate and gives an approximation. This kind of algorithm is described in the next section.     

\subsection{Randomized PCA algorithm}

In order to make computing PCs faster, we can use randomized version of algorithm. The full method is presented in Algorithm \ref{alg:rpca1}. 

\begin{algorithm}
    \caption{Randomized PCA}\label{alg:rpca1}
    \begin{algorithmic}[1]
    \Function{RPCA}{$k$, $X \in \mathbb{R}^{m \times n}$} 
        \State $(p,q) \leftarrow \text{(20,1)}$
        \State $Q \leftarrow \text{randn(n, k+p)}$
            \For{$i \in \{1,\cdots,q\}$}
            \State $\psi \leftarrow X^TXQ$
            \State $Q \leftarrow \text{orthogonalize($\psi$)}$
            \EndFor
            \State $F \leftarrow (X^TXQ)^T(X^TXQ)$
            \State $(V,\Sigma^2) \leftarrow \text{eig($F$,$k$)}$
            \State $V \leftarrow (X^TXQ)V\Sigma^P$
            \State $\text{return } (V, \Sigma)$
        \EndFunction
    \end{algorithmic}
\end{algorithm}

The algorithm takes two parameters on its input: data matrix $X \in \mathbb{R}^{m \times n}$ and input parameter $k$, which is responsible for compression degree (number of eigenvectors that are considered). The most imortant idea in this approach is to find the range for $X^TX$ matrix. 
The range for matrix $A$ can be understood as a collection of vectors $b$, which satisfy the equation $Ax=b$. Let us notice that eigenvectors might appear frequently as solutions of this equation. In fact, the range of $\psi$ will tend to be more aligned with the space spanned by the top eigenvectors of $X^TX$ \cite{Mineiro}. 

The randomized range finder begins in 4. line and computes an orthogonal basis for $\psi$. This operation is repeated $q$ times. We can certainly manipulate a parameter of $q$, however is should be rather small, beacues the complexity of computing an orthogonal basis is $O(nk^2)$. 
After finding a good aproximation for the principal subspace of $X^TX$, we optimize fully over that subspace and back out the solution. It is worth emphasising that the last steps, including eigendecoposition of $F \in \mathbb{R}^{(k+p) \times (k+p)}$, are cheap. It is caused by low dimensions of $F$ matrix \cite{Mineiro}.

We can now use the Randomized PCA algorithm with linear regressor, what is shown by Algorithm \ref{alg:rpca2} 

\begin{algorithm}
    \caption{Multi-dimension linear regressor with preliminary data reduction by RPCA}\label{alg:rpca2}
    \begin{algorithmic}[1]
    \State Let $X^{*}$ be a copy of $X$ and $Y^{*}$ be a copy of $Y$
    \If{reduceX == true}
    \State $(V_{x}, \Sigma_{x}) = RPCA(X,h1)$
        \State $X^{*}=X*V_{x}$
    \EndIf
    \If{reduceY == true}
        \State $(U_{y}, \Sigma_{y}) = RPCA(Y,h2)$
        \State $Y^{*}=Y*V_{y}$
    \EndIf
    \State Learn linear regressor $r(X^*)$ from $\{X^*,Y^*\}$
    \item[]
    \State Let $x^*$ be a copy of instance $x$
    \If{reduceX == true}
        \State $x^* = x^*V_x$
    \EndIf
    \If{reduceY == true}
        \State Predict the label-set of $x^*$ by $h(x^*)=round( r(x^*) * V_y^T)$ 
    \Else
        \State Predict the label-set of $x^*$ by $h(x^*)=round( r(x^*) )$ 
    \EndIf

    \end{algorithmic}
\end{algorithm}


\section{Label space reduction inspired by CCA}

Multi-label classifiers based on preliminary PCA compression of data have an essential disadventage - the reduction of label (feature) space is made independently of feature (label) space. In other words, label space is not aware of reduction feature space. One of possibilities, to deal with this issue, is to use statistical method called Canonical Correlation Analysis or rather methods inspired by CCA.   

\subsection{Canonical correlation analysis}

Let us assume you are given two data sets (matrices): $X \in \mathbb{R}^{m \times n_1}$ and $Y \in \mathbb{R}^{m \times n_2}$. Without loss of generality you suppose that we have substracted the mean in each column. CCA aims at finding linear combination $x$ coordinates that correlates well over the data with an linear combination of the $y$. In other words, you want to find best matched pair of linear combination of $X$ and $y$, which have the largest coefficent of correlation. You certainly do not need to stop there - you can ask for second-best pair, third-best pair and so on \cite{William}.

Formally, this problem can be expressed by two equations:
\begin{equation}\label{eq:cca1}
    U=XW_x^T,   V=YW_y^T    
\end{equation}
where $W_x \in \mathbb{R}^{d \times n_1}$, $W_y \in \mathbb{R}^{d \times n_2}$ are matrices of coefficents of linear combinations. $U \in \mathbb{R}^{m \times d}$ and $V \in \mathbb{R}^{m \times d}$ certainly are matrices of evaluated linear combinations. $d=min(rank(X), rank(Y))$ is the number o pairs, where $rank$ means column rank. Let us denote i-th rows of $W_x$, $W_y$, $U$ and $V$ by: $w_x^{(i)}$, $w_y^{(i)}$, $u^{(i)}$ and $v^{(i)}$. For $i=1,\cdots,d$ the following condition must be met \cite{William}:
\begin{equation}\label{eq:cca2}
    (w_x^{(i)}, w_y^{(i)})=\argmax_{w_x^{(i)}, w_y^{(i)}} corr[X(w_x^{(i)})^T, Y(w_y^{(i)})^T] = \argmax_{w_x^{(i)}, w_y^{(i)}} \left[\frac{u^{(i)}(v^{(i)})^T}{\sqrt{u^{(i)}(u^{(i)})^T}\sqrt{v^{(i)}(v^{(i)})^T}}\right] 
\end{equation}
You should also remember that the correlation betweeen $u^{(i)}$ and $v^{(j)}$ where $i\neq j$ is equal to $0$.

Kettenring showed that CCA is equivalent to simultaneously solving the following constrained optimization problem \cite{ChenLin}:
\begin{equation}\label{eq:cca3}
\begin{split}
    \min_{W_x, W_y} \norm{U-V}_{F}^2 = \min_{W_x, W_y} \norm{XW_x^T-YW_y^T}_{F}^2 \\ 
    \text{   subject to   } W_xX^TXW_x^T=W_yY^TYW_y^T=I   
\end{split}
\end{equation}

From the machine learning point of view, you can treat $X$ and $Y$ matrices as feature and label matrices and perform CCA on them. Then you can obtain $U$ and $V$ matrices, which certainly are linear combinations of features and suitably labels in this case. The final step is to learn linear regressor from $U$ to $V$. Unfortunately, this kind of algorithm is time-consuming. Let us notice that encoding and decoding data operations are slow, because $W_x$ and $W_y$ matrices are not orthogonal (you cannot simply invert those matrices). Although there is no point in using pure CCA to build classifier, you can use methods which base on CCA concept. This kind of algorithm will be discussed in the next section.

\subsection{CPLST algorithm}

CPLST algorithm stands for \textit{Conditional Prinicipal Label Space Transformation}, which means that only label space will be transformed and then reduced. In contrast to PCA approach, this kind of reduction is feature-aware. 

As it has been already mentioned, CCA can be considered as an optimization problem of finding minimal prediction error (equation \ref{eq:cca3}) under the constraint $W_xX^TXW_x^T=W_yY^TYW_y^T=I$. As long as you take into account only label space, you can drop $W_xX^TXW_x^T=I$ constraint, as it is connected with feature space, which is not transformed. It is also worth enforcing orthogonalization of $W_y$ matrix, which simplifies decoding of data. As a result you obtain the following problem:

\begin{equation}\label{eq:cplst1}
    \min_{W_x, W_y} \norm{XW_x^T-YW_y^T}_{F}^2  
    \text{  subject to   } W_yW_y^T=I   
\end{equation}
Let us notice that finding $W_x$ is just linear regression from $X$ to $YW_y^T$. It means that optimal $W_x$ is determined by following equation:
\begin{equation}\label{eq:cplst2}
\begin{split}
    XW_x^T=YW_y^T \\
    W_x^T=X^PYW_y^T
\end{split}
\end{equation}
When optimal $W_x$ is inserted back into (\ref{eq:cplst1}), the optimization problem becomes:
\begin{equation}\label{eq:cplst3}
    \min_{W_yW_y^T=I} \norm{XX^PYW_y^T-YW_y^T}_{F}^2=\min_{W_yW_y^T=I} \norm{(XX^P-I)YW_y^T}_{F}^2
\end{equation}
For every matrix A $\norm{A}_{F}^2=tr(A^TA)$, so above issue is equivalent to:
\begin{equation}\label{eq:cplst4}
    \min_{W_yW_y^T=I} tr(W_yY^T(I-H)YW_y^T)
\end{equation}
Matrix $H=XX^P$ is the hat matrix for linear regression. The presented approach is called \textit{Orthogonally constraint CCA}. In order to solve (\ref{eq:cplst4}), you can consider the eigenvectors that correspond to the largest eigenvalues of $Y^T(H-I)Y$ (it can be reached by SVD decomposition) \cite{ChenLin}.

As you see, OCCA aims at minimizing the prediction error (\ref{eq:cplst1}), however an encoding error of label space is not taken into account. In fact, label space is transformed by eigenvectors captured from $Y^T(H-I)Y$ matrix. If you combine OCCA approach with minimizing the encoding error, then you get CPLST algorithm. It can be described by following formula:

\begin{equation}\label{eq:cplst5}
    \min_{W_x, W_y} \left(\norm{XW_x^T-YW_y^T}_{F}^2 + \norm{Y-YW_y^TW_y}_{F}^2\right)  
    \text{  subject to   } W_yW_y^T=I   
\end{equation}
If you use again (\ref{eq:cplst2}) and relation between trace of matrix and its Frobenius norm, then you get:
\begin{equation}\label{eq:cplst6}
    \min_{W_yW_y^T} tr(W_yY^T(I-H)YW_y^T-W_y^TW_yY^TY-Y^TYW_y^TW_y+W_y^TW_yY^TYW_y^TW_y)  
\end{equation}
After eliminating a pair of $W_y$ and $W_y^T$ by cyclic permutation and combining the last three terms of (\ref{eq:cplst6}), you get:

\begin{equation}\label{eq:cplst6}
    \max_{W_yW_y^T} tr(W_yY^THYW_y^T)  
\end{equation}

The above problem can be solved analogously to OCCA - by finding eigenvectors and eigenvalues of $Y^THYT$ The resulting algorithm is presented in Algorithm \ref{alg:cplst1} \cite{ChenLin}. Besides balance between the prediction error and the encoding error, CPLST is faster than OCCA. It is an effect of decomposing $Z^THZ$ matrix instead of $Z^T(H-I)Z$. On the other hand, both methods are slower than algorithms based on PCA reduction. It is caused by calculating $H$ matrix, which is time-consuming for large input data.

\begin{algorithm}
    \caption{Conditional Principal Label Space Transformation}\label{alg:cplst1}
    \begin{algorithmic}[1]
        \State Let $Z=[z_1 \cdots z_m]^T$ with $z_i=y_i-\bar{y}$
        \State Perform SVD on $Z^THZ$ to obtain $Z^THZ=U\Sigma V^T$ with $\sigma_{1} \geq \sigma_{2} \geq \cdots \sigma_{m}$. Let $V_h$ contain the left $h$ columns of $V$
        \State Encode $\{(x_i,z_i)\}^{m}_{i=1}$ to $\{(x_i,t_i)\}^{m}_{i=1}$ where $t_i=V^T_hz_i$
        \State Learn a multi-dimension regressor $r(x)$ from $\{(x_i,t_i)\}^{m}_{i=1}$ 
        \State Predict the label-set of an instance $x$ by $h(x)=round(V_hr(x)+\bar{y})$  
    \end{algorithmic}
\end{algorithm}



\chapter{Implementation}

\section{Requirements}

\subsection{General information}
All the methods, which have been discussed in this paper, are a part of a library that can be used to solve multi-label classification problems in practise. The library was called \textit{MLCPACK}, what stands for \textit{Multi-Label Classification PACKage}. The \textit{API} is implemented in \textit{C++11} programming language and should be compiled by \textit{GCC ($>=$4.9)} in order to support all of the new \textit{C++} features. The other compilers (Clang, Visual C++ Compiler etc.) were not checked, so there is no warranty the build process succeeds in those cases.  The compilation process in managed by \textit{CMake} tool, which is cross-platform and can be used in various operating systems \cite{CMake}. However, the library was tested only in \textit{Linux} environment (Debian 8.2). 

Besides using standard \textit{C++} library, there are also two external libraries which must be installed in a system: 

\begin{itemize}
    \item Boost ($>=$1.55)
    \item Armadillo ($>=$6.100)
\end{itemize}
The first one contains useful modules which extends possibilities of standard library (unittests, file system etc.), while \textit{Armadillo} is an advanced linear algebra library. More details connected to build process is presented in the Appendix \ref{app:build}. 


\subsection{Discussing linear algebra library used in project}

The efficency of the algorithms, which are presented in Chapter 2, is strongly dependent on an implementations of the linear algebra operations such as pseudoinverse, SVD decomposition etc. After analysing available solutions, the \textit{API} provided by the \textit{Armadillo} was chosen. The \textit{Armadillo} is high-quality library dedicated for \textit{C++} developers. The main strength of it is a good balance between speed and ease of use. In fact, its syntax is similar to \textit{Matlab} environment. The example of code is presented in the Appendix \ref{app:arma}.

The usage of Armadillo is simple, as it is shawn in the Appendix \ref{app:arma}. However, the most important issue is a real speed of matrix operations. According to the documentation, the \textit{API} is integrated with \textit{LAPACK} and \textit{BLAS}, what means that effectiveness of matrix multiplication or its decompositions are dependent on their implementations. These libraries are known as rather fast. There is also a possibility of linking \textit{OpenBLAS} instead of standard \textit{BLAS}. \textit{OpenBLAS} supports multithreading. The number of threads, involved in computation, can be simply controlled by setting specific environment variables, i.e. \textit{OPENBLAS\_NUM\_THREADS} \cite{Blas}. In the implementation of \textit{MLCPACK}, \textit{OpenBLAS} library is used.    

\begin{figure}[h]
\centering
\caption{Multiplication matrix performance}
\label{fig:mulperf}
\begin{tikzpicture}
    \begin{axis}[legend pos=north east,
        xlabel={size of matrix},
        ylabel={time [s]},
        xmin=0,
        xmax=10500,
        ymin=0,
        ymax=200000,
        ymode=log]
        \addplot table [x=size, y=arma, col sep=semicolon] {figures/mul_test.csv};
        \addlegendentry{Armadillo}
        \addplot table [x=size, y=numpy, col sep=semicolon] {figures/mul_test.csv};
        \addlegendentry{NumPy}
    \end{axis}
\end{tikzpicture}
\end{figure}

\begin{figure}
\centering
\caption{SVD decomposition of matrix performance}
\label{fig:svdperf}
\begin{tikzpicture}
    \begin{axis}[legend pos=north east,
        xlabel={size of matrix},
        ylabel={time [s]},
        xmin=0,
        xmax=4500,
        ymin=0,
        ymax=650]
        \addplot table [x=size, y=arma, col sep=semicolon] {figures/svd_test.csv};
        \addlegendentry{Armadillo}
        \addplot table [x=size, y=numpy, col sep=semicolon] {figures/svd_test.csv};
        \addlegendentry{NumPy}

    \end{axis}
\end{tikzpicture}
\end{figure}


Figure \ref{fig:mulperf} and Figure \ref{fig:svdperf} show the speed of basic matrix operations performed by \textit{Armadillo} and \textit{NumPy} which is analogous linear algebra module for \textit{Python} programming language. As we see \textit{Armadillo} is definitely faster than \textit{NumPy} for matrix multiplication and \textit{SVD} decomposition as well.   
The experiment was run on the machine which has Intel Core i3-350M 2.26 GHz processor and 4096 MB of RAM. In case of \textit{Armadillo} test, the compilation of program was optimized (third level of optimization) and \textit{OpenBLAS} was used instead of standard \textit{BLAS} library. \textit{NumPy} (1.8.2) was tested in \textit{Python3.4} environment (standard installation from \textit{Debian} repository). The diffrences in time-consumption justify the choice of \textit{Armadillo} and \text{C++11} programming language, however \textit{NumPy} also uses \textit{BLAS} what means that it is possible to install \textit{NumPy} with \textit{OpenBLAS} support.

\section{Structure of library}

\textit{Mlcpack} consists off two main modules:
\begin{itemize}
\item Algorithms
\item Utils
\end{itemize}

\subsection{Description of Algorithms module}

The Algorithms module contains all implementations of algorithms discussed in this paper. Its building is based on \textit{strategy} design pattern which is typical for specific family of algorithms. A client is allowed to simply rotate various methods - in other words, algorithms are replaced regardless of clients which use them. The structure of this module is shown in Figure \ref{fig:alg_sh}. 

The base class for algorithm implementations is called \textit{IStrategy} and contains two pure virtual methods: \textit{learn} and \textit{classify} which must be implemented. The second function takes an object of \textit{Intance} type as an argument which certainly wraps a particular instance, and returns binary vector of labels. The constructor of \textit{IStrategy} takes an only one argument of type \textit{Instances} which represents a training set.  Let us notice that \textit{LinearRegression} class, which inherits directly from \textit{IStrategy}, is a base class for the rest of methods. The regressor is an important component for all the approaches, therefore this relation is represented by an inheritance. In \textit{LinearRegression}, there are also two additonal methods: \textit{save} and \textit{load} which take one argument of \textit{string} type. These functions are responsible for serialization of a trained classifier. The idea is to store such classifier on hard drive and load it into memory if it is needed. It certainly allows to save time, ecspecially if we take into account massive training sets. In contrast to \textit{IStrategy} interface, \textit{LinearRegression} class and its derived classes have more complex constructors. The linear regressor class constructor(and its derived classes constructors as well) takes an additional argument which is the regularization parameter. \textit{CPLST}, \textit{LRWithPCA} etc. classes constructors need also the reduction degree parameter. The "learning flow" is presented in Appendix \ref{app:learning}.

The Algorithms module also contains class (\textit{Evaluation}) which is used to evaluate particular methods. It allows to measure quality of classification by parameters, such as Micro-, Instance- and Macro-average of Precision, Recall and F-score. It also calculates Hamming Loss metric. The \textit{API} is simple, as it is shown in Figure \ref{fig:alg_sh}. To construct an evaluator, we have to pass two arguments: a training set and a number of folds used for cross-validation purpose. In order to start the whole process, we certainly invoke \textit{evaluate} method which takes a reference to an algorithm object. The rest of available functions allow to get a specific evaluation metric. The "evaluating flow" is presented in Appendix \ref{app:evaluating}.

\begin{figure}
\centering
\caption{UML diagram of Algorithms module}
\label{fig:alg_sh}
\includegraphics[scale=0.5]{figures/mlcpack.png}
\end{figure}

\subsection{Description of Utils module}

The schema of \textit{Utils} module is shown in Figure \ref{fig:utils_sh}. The module contain classes which process and provide training data for the "algorithm" objects. The \textit{API} was designed to support \textit{ARFF} data format (an example of data written in this format is presented in the Appendix \ref{app:arff}), however there is a possibility to add different types of parsers. Each parser must inherit from \textit{IParser} interface and implement two methods: \textit{parse} and \textit{getInstances}. The first method is certainly responsible for processing an input file, while the second method wraps parsed, raw data by \textit{Instance} and \textit{Instances} objects. Let us notice that \textit{ArffParser} objects take an additional argument in its constructor. This second argument is a path to a file containing extra information about attributes which are, in fact, labels.

\textit{Instance} and \textit{Instances} classes, which have already been mentioned, are used to simplify making operations on data. An object of \textit{Instance} type wraps a single instance in a data set. As we see in Figure \ref{fig:utils_sh}, this class contains methods which allow to write (\textit{setAttributeValue}) or read (\textit{getValueOfAttr}) value of a specific attribute of an instance. An object of \textit{Instances} type is a collection of \textit{Instance} objects. \textit{getAttributeMat} and \textit{getTargetsMat} methods of this class are used to generate the matrix of features and the matrix of labels as objects provided by \textit{Armadillo} library. There is also \textit{shuffle} method which changes randomly the order of instances.  

\begin{figure}
\centering
\caption{UML diagram of Utils module}
\label{fig:utils_sh}
\includegraphics[scale=0.5]{figures/Utils.png}
\end{figure}



\chapter{Results}

\section{Experiment environment}
\subsection{Data sets used in experiment}

All of the data sets used in the experiment come from \textit{Mulan} project web page. The sizes of sets are differentiated in order to check behaviours of the algorithms for a low, a medium and a high number of instances. The detailed information about data sets are presented in \Cref{tab:exp1}. 

\begin{table}[h]
\centering
\caption{Statistic and details of data sets used in the experiment}
\label{tab:exp1}
    \begin{tabular}{l|c|r|r|r}
    name & domain & instances & features & labels \\ \hline \hline
    enron & text  &  1702  & 1001 &  53 \\
    scene & image &  2407  & 294 & 6 \\   
    yeast & biology & 2417 & 103 & 14 \\
    bibtex & audio & 7395 & 1836 & 159 \\
    corel16k (10 samples) & images & 13811 $\pm$ 87 & 500 & 161 $\pm$ 9\\
    EUR-Lex (directory codes) & text & 19348 & 5000 & 412 \\
    bookmarks & text & 87856 & 2150 & 208
    \end{tabular}
\end{table}

\subsection{Parameters of desktop}

The experiment was performed on one of the available instances (m4.4xlarge) provided by \textit{Amazon Web Services}. It has 64 GB of memory and \textit{Intel Xeon® E5-2676 v3 (Haswell)} processor. Some details connected with CPU can be found in \Cref{tab:cpu}. The operating system, used in the experiment, was \textit{Ubuntu 14}.

\begin{table}[h]
\centering
\caption{Extra information about Intel Xeon® E5-2676 v3 (Haswell)}
\label{tab:cpu}
    \begin{tabular}{l|l}
    & \\ \hline \hline
    Cache & 20 MB SmartCache \\
    Bus speed & 8 GT/s QPI \\
    Instruction set & 64-bit \\
    Number of cores & 8 \\
    Number of threads & 16 \\
    Processor base frequency & 2.4 GHZ \\
    Intel$^{®}$ Turbo Boost Technology$^{‡}$ & 2.0 \\
    Intel$^{®}$ Turbo vPro Technology$^{‡}$ & yes \\ 
    Intel$^{®}$ Turbo Hyper-Threading Technology$^{‡}$ & yes \\
    Intel$^{®}$ Turbo Virtualization Technology$^{‡}$ & yes
    \end{tabular}
\end{table}

The library was compiled by GCC 4.9 with the third optimization level (\textit{-O3} flag). The build process certainly takes more time and memory due to inlining functions or loop unrolling etc., however a performance of code is faster. We can find more information about the optimization flags in GCC documentation\footnote{\bibentry{Opt}}. It is also worth mentioning that OpenBLAS was used instead of standard BLAS API in order to enhance the efficiency of linear algebra operations and to use multithreading. 

\section{Comparison of proposed methods}
\subsection{Metrics of accuracy}

The accuracy of the algorithms was measured by the standard metrics, characteristic for multi-label classification.

\subsubsection{Precision and recall} 

\textit{Precision} and \textit{recall} are very similar measures. Precision is a fraction of a number of correct labels to a number of predicted labels, while recall (also called \textit{sensitivity}) is a fraction of a number of correct labels to a number of targets. They can be simply described by the following formulas:  
\begin{equation}
\label{eq:exp2}
precision=\frac{|T \cap P|}{|P|}
\end{equation}
\begin{equation}
\label{eq:exp3}
recall=\frac{|T \cap P|}{|T|}
\end{equation}
where $T$ is a set of targets, while $P$ is a set of predictions. Although their definitions are always the same, they can be computed by three different methods:
\begin{itemize}
\item Macro-average - in this method we compute precision and recall for each category independently. Finally we take the arithmetic average of these values.
\item Micro-average - this method is the most intuitive. We simply sum up true positives, false positives and false negatives for a whole set and then apply them to get the metric. 
\item Instance-average - in this approach we compute precision and recall for each instance. In the next step, similarly to Macro-average, the arithmetic average is calculated.
\end{itemize}

Unfortunately none of these methods is more significant than the others, so the best way is to use every approach in order to compare the accuracy of classifiers.  

\subsubsection{F1 score}

This metric is simply the harmonic average of Precision and Recall. Its value is certainly dependent on a chosen method of calculating Precision and Recall. As a result, we define Micro-, Macro-, Instance-average F1 score. 

\subsubsection{Hamming loss}

This metric\footnote{\bibentry{Loss}} can be defined as a fraction of number of wrong labels to a total number of labels. It is described by the following formula:
\begin{equation}
\label{eq:exp1}
    HammingLoss(z_i, y_i)=\frac{1}{m}\sum\limits_{i=1}^{m}\frac{xor(z_i,y_i)}{k}
\end{equation}
where $m$ is a number of instances, $k$ is a number of labels, $y_{i}$ is the ground truth and $z_{i}$ is the prediction. Let you notice that unlike the other metrics, we want Hamming loss to be as low as possible.  

\subsection{Comparison of methods accuracy}

The results of accuracy experiment are divided into sections. Each section is connected with a particular data set. The experiment involves the following algorithms:
\begin{itemize}
    \item CPLST,
    \item LR,
    \item OCCA,
    \item LRWithPCA which denotes linear regression with PCA transformation on a label space,
    \item LRWithRandomPCA which denotes linear regression with randomized PCA compression on a label space.
\end{itemize}
All of the tested approaches used a linear regressor with Tikhonov regularization. The regularization coefficient was equal to $7.0$. This value was certainly stated experimentally and for different data sets it does not have to be optimal.  

\Crefrange{tab:exp1}{tab:exp7} contain scores of Micro-, Macro-, Instance-average F1 score metrics for particular methods. In case of the algorithms with data compression, an only label space was reduced. The reduction degree was equal to $0.6$ what means that $40\%$ of columns in a transformed space were rejected.
\Crefrange{fig:enron}{fig:bookmarks} show the influence of a value of reduction degree on a specific metric. 

\newpage
\subsubsection{Enron data set}

\begin{table}[H]
\centering
\caption{Accuracy of methods for enron data set}
\label{tab:exp1}
\pgfplotstabletypeset [
    col sep=semicolon,
    columns/Method/.style={string type, column type={c|}},
    columns/Macro F1 score/.style={column type={c|}},
    columns/Micro F1 score/.style={column type={c|}},
    columns/Instance F1 score/.style={column type={c|}},
    columns/Hamming loss/.style={column type={c|}},
    every head row/.style={after row=\hline\hline}
]{figures/enron.csv}
\end{table}

\begin{figure}[H]
%\centering
\caption{Relation between reduction degree and accuracy for enron data set}
\label{fig:enron}

\begin{subfigure}{.5\textwidth}
\caption{Hamming loss}
\label{fig:exp1}
\begin{tikzpicture}
    \begin{axis}[legend pos=north east,
        legend style={font=\fontsize{7}{5}\selectfont}, 
        xlabel={Degree of reduction},
        ylabel={Hamming loss},
        width=7cm,
        xmax=1,
        ymax=0.07]
        \addplot table [x=k, y=LRWithPCA, col sep=semicolon] {figures/enron_0.csv};
        \addlegendentry{LRWithPCA}
        \addplot table [x=k, y=CPLST, col sep=semicolon] {figures/enron_0.csv};
        \addlegendentry{CPLST}
        \addplot table [x=k, y=LRWithRandomPCA, col sep=semicolon] {figures/enron_0.csv};
        \addlegendentry{LRWithRandomPCA}
        \addplot table [x=k, y=LR, col sep=semicolon] {figures/enron_0.csv};
        \addlegendentry{OCCA}
        \addplot table [x=k, y=OCCA, col sep=semicolon] {figures/enron_0.csv};
        \addlegendentry{LR}

    \end{axis}
\end{tikzpicture}
\end{subfigure}
\begin{subfigure}{.5\textwidth}
\caption{Macro F1 score}
\label{fig:exp2}
\begin{tikzpicture}
    \begin{axis}[legend pos=north east,
        legend style={font=\fontsize{7}{5}\selectfont}, 
        xlabel={Degree of reduction},
        ylabel={Macro F1 score},
        width=7cm,
        xmax=1,
        ymax=0.35]
        \addplot table [x=k, y=LRWithPCA, col sep=semicolon] {figures/enron_1.csv};
        \addlegendentry{LRWithPCA}
        \addplot table [x=k, y=CPLST, col sep=semicolon] {figures/enron_1.csv};
        \addlegendentry{CPLST}
        \addplot table [x=k, y=LRWithRandomPCA, col sep=semicolon] {figures/enron_1.csv};
        \addlegendentry{LRWithRandomPCA}
        \addplot table [x=k, y=OCCA, col sep=semicolon] {figures/enron_1.csv};
        \addlegendentry{OCCA}
        \addplot table [x=k, y=LR, col sep=semicolon] {figures/enron_1.csv};
        \addlegendentry{LR}

    \end{axis}
\end{tikzpicture}
\end{subfigure}

\begin{subfigure}{.5\textwidth}
\caption{Micro F1 score}
\label{fig:exp3}
\begin{tikzpicture}
    \begin{axis}[legend pos=north east,
        legend style={font=\fontsize{7}{5}\selectfont}, 
        xlabel={Degree of reduction},
        ylabel={Micro F1 score},
        width=7cm,
        xmax=1,
        ymax=0.65]
        \addplot table [x=k, y=LRWithPCA, col sep=semicolon] {figures/enron_2.csv};
        \addlegendentry{LRWithPCA}
        \addplot table [x=k, y=CPLST, col sep=semicolon] {figures/enron_2.csv};
        \addlegendentry{CPLST}
        \addplot table [x=k, y=LRWithRandomPCA, col sep=semicolon] {figures/enron_2.csv};
        \addlegendentry{LRWithRandomPCA}
        \addplot table [x=k, y=OCCA, col sep=semicolon] {figures/enron_2.csv};
        \addlegendentry{OCCA}
        \addplot table [x=k, y=LR, col sep=semicolon] {figures/enron_2.csv};
        \addlegendentry{LR}

    \end{axis}
\end{tikzpicture}
\end{subfigure}
\begin{subfigure}{.5\textwidth}
\caption{Instance F1 score}
\label{fig:exp4}
\begin{tikzpicture}
    \begin{axis}[legend pos=north east,
        legend style={font=\fontsize{7}{5}\selectfont},
        xlabel={Degree of reduction},
        ylabel={Instance F1 score},
        width=7cm,
        xmax=1,
        ymax=0.7]
        \addplot table [x=k, y=LRWithPCA, col sep=semicolon] {figures/enron_3.csv};
        \addlegendentry{LRWithPCA}
        \addplot table [x=k, y=CPLST, col sep=semicolon] {figures/enron_3.csv};
        \addlegendentry{CPLST}
        \addplot table [x=k, y=LRWithRandomPCA, col sep=semicolon] {figures/enron_3.csv};
        \addlegendentry{LRWithRandomPCA}
        \addplot table [x=k, y=OCCA, col sep=semicolon] {figures/enron_3.csv};
        \addlegendentry{OCCA}
        \addplot table [x=k, y=LR, col sep=semicolon] {figures/enron_3.csv};
        \addlegendentry{LR}
    \end{axis}
\end{tikzpicture}
\end{subfigure}

\end{figure}

\subsubsection{Scene data set}

\begin{table}[H]
\centering
\caption{Accuracy of methods for scene data set}
\label{tab:exp2}
\pgfplotstabletypeset [
    col sep=semicolon,
    columns/Method/.style={string type, column type={c|}},
    columns/Macro F1 score/.style={column type={c|}},
    columns/Micro F1 score/.style={column type={c|}},
    columns/Instance F1 score/.style={column type={c|}},
    columns/Hamming loss/.style={column type={c|}},
    every head row/.style={after row=\hline\hline}
]{figures/scene.csv}
\end{table}

\begin{figure}[H]
%\centering
\caption{Relation between reduction degree and accuracy for scene data set}
\label{fig:scene}

\begin{subfigure}{.5\textwidth}
\caption{Hamming loss}
\label{fig:exp5}
\begin{tikzpicture}
    \begin{axis}[legend pos=north east,
        legend style={font=\fontsize{7}{5}\selectfont}, 
        xlabel={Degree of reduction},
        ylabel={Hamming loss},
        width=7cm,
        xmax=1,
        ymax=0.35]
        \addplot table [x=k, y=LRWithPCA, col sep=semicolon] {figures/scene_0.csv};
        \addlegendentry{LRWithPCA}
        \addplot table [x=k, y=CPLST, col sep=semicolon] {figures/scene_0.csv};
        \addlegendentry{CPLST}
        \addplot table [x=k, y=LRWithRandomPCA, col sep=semicolon] {figures/scene_0.csv};
        \addlegendentry{LRWithRandomPCA}
        \addplot table [x=k, y=OCCA, col sep=semicolon] {figures/scene_0.csv};
        \addlegendentry{OCCA}
        \addplot table [x=k, y=LR, col sep=semicolon] {figures/scene_0.csv};
        \addlegendentry{LR}
    \end{axis}
\end{tikzpicture}
\end{subfigure}
\begin{subfigure}{.5\textwidth}
\caption{Macro F1 score}
\label{fig:exp6}
\begin{tikzpicture}
    \begin{axis}[legend pos=north east,
        legend style={font=\fontsize{7}{5}\selectfont}, 
        xlabel={Degree of reduction},
        ylabel={Macro F1 score},
        width=7cm,
        xmax=1,
        ymax=1.5]
        \addplot table [x=k, y=LRWithPCA, col sep=semicolon] {figures/scene_1.csv};
        \addlegendentry{LRWithPCA}
        \addplot table [x=k, y=CPLST, col sep=semicolon] {figures/scene_1.csv};
        \addlegendentry{CPLST}
        \addplot table [x=k, y=LRWithRandomPCA, col sep=semicolon] {figures/scene_1.csv};
        \addlegendentry{LRWithRandomPCA}
        \addplot table [x=k, y=OCCA, col sep=semicolon] {figures/scene_1.csv};
        \addlegendentry{OCCA}
        \addplot table [x=k, y=LR, col sep=semicolon] {figures/scene_1.csv};
        \addlegendentry{LR}
    \end{axis}
\end{tikzpicture}
\end{subfigure}

\begin{subfigure}{.5\textwidth}
\caption{Micro F1 score}
\label{fig:exp7}
\begin{tikzpicture}
    \begin{axis}[legend pos=north east,
        legend style={font=\fontsize{7}{5}\selectfont}, 
        xlabel={Degree of reduction},
        ylabel={Micro F1 score},
        width=7cm,
        xmax=1,
        ymax=1.5]
        \addplot table [x=k, y=LRWithPCA, col sep=semicolon] {figures/scene_2.csv};
        \addlegendentry{LRWithPCA}
        \addplot table [x=k, y=CPLST, col sep=semicolon] {figures/scene_2.csv};
        \addlegendentry{CPLST}
        \addplot table [x=k, y=LRWithRandomPCA, col sep=semicolon] {figures/scene_2.csv};
        \addlegendentry{LRWithRandomPCA}
        \addplot table [x=k, y=OCCA, col sep=semicolon] {figures/scene_2.csv};
        \addlegendentry{OCCA}
        \addplot table [x=k, y=LR, col sep=semicolon] {figures/scene_2.csv};
        \addlegendentry{LR}
    \end{axis}
\end{tikzpicture}
\end{subfigure}
\begin{subfigure}{.5\textwidth}
\caption{Instance F1 score}
\label{fig:exp8}
\begin{tikzpicture}
    \begin{axis}[legend pos=north east,
        legend style={font=\fontsize{7}{5}\selectfont},
        xlabel={Degree of reduction},
        ylabel={Instance F1 score},
        width=7cm,
        xmax=1,
        ymax=1.3]
        \addplot table [x=k, y=LRWithPCA, col sep=semicolon] {figures/scene_3.csv};
        \addlegendentry{LRWithPCA}
        \addplot table [x=k, y=CPLST, col sep=semicolon] {figures/scene_3.csv};
        \addlegendentry{CPLST}
        \addplot table [x=k, y=LRWithRandomPCA, col sep=semicolon] {figures/scene_3.csv};
        \addlegendentry{LRWithRandomPCA}
        \addplot table [x=k, y=OCCA, col sep=semicolon] {figures/scene_3.csv};
        \addlegendentry{OCCA}
        \addplot table [x=k, y=LR, col sep=semicolon] {figures/scene_3.csv};
        \addlegendentry{LR}
    \end{axis}
\end{tikzpicture}
\end{subfigure}


\end{figure}


\subsubsection{Yeast data set}

\begin{table}[H]
\centering
\caption{Accuracy of methods for yeast data set}
\label{tab:exp3}
\pgfplotstabletypeset [
    col sep=semicolon,
    columns/Method/.style={string type, column type={c|}},
    columns/Macro F1 score/.style={column type={c|}},
    columns/Micro F1 score/.style={column type={c|}},
    columns/Instance F1 score/.style={column type={c|}},
    columns/Hamming loss/.style={column type={c|}},
    every head row/.style={after row=\hline\hline}
]{figures/yeast.csv}
\end{table}

\begin{figure}[H]
%\centering
\caption{Relation between reduction degree and accuracy for yeast data set}
\label{fig:yeast}

\begin{subfigure}{.5\textwidth}
\caption{Hamming loss}
\label{fig:exp9}
\begin{tikzpicture}
    \begin{axis}[legend pos=north east,
        legend style={font=\fontsize{7}{5}\selectfont}, 
        xlabel={Degree of reduction},
        ylabel={Hamming loss},
        width=7cm,
        xmax=1,
        ymax=0.25]
        \addplot table [x=k, y=LRWithPCA, col sep=semicolon] {figures/yeast_0.csv};
        \addlegendentry{LRWithPCA}
        \addplot table [x=k, y=CPLST, col sep=semicolon] {figures/yeast_0.csv};
        \addlegendentry{CPLST}
        \addplot table [x=k, y=LRWithRandomPCA, col sep=semicolon] {figures/yeast_0.csv};
        \addlegendentry{LRWithRandomPCA}
        \addplot table [x=k, y=OCCA, col sep=semicolon] {figures/yeast_0.csv};
        \addlegendentry{OCCA}
        \addplot table [x=k, y=LR, col sep=semicolon] {figures/yeast_0.csv};
        \addlegendentry{LR}
    \end{axis}
\end{tikzpicture}
\end{subfigure}
\begin{subfigure}{.5\textwidth}
\caption{Macro F1 score}
\label{fig:exp10}
\begin{tikzpicture}
    \begin{axis}[legend pos=north east,
        legend style={font=\fontsize{7}{5}\selectfont}, 
        xlabel={Degree of reduction},
        ylabel={Macro F1 score},
        width=7cm,
        xmax=1,
        ymax=0.5]
        \addplot table [x=k, y=LRWithPCA, col sep=semicolon] {figures/yeast_1.csv};
        \addlegendentry{LRWithPCA}
        \addplot table [x=k, y=CPLST, col sep=semicolon] {figures/yeast_1.csv};
        \addlegendentry{CPLST}
        \addplot table [x=k, y=LRWithRandomPCA, col sep=semicolon] {figures/yeast_1.csv};
        \addlegendentry{LRWithRandomPCA}
        \addplot table [x=k, y=OCCA, col sep=semicolon] {figures/yeast_1.csv};
        \addlegendentry{OCCA}
        \addplot table [x=k, y=LR, col sep=semicolon] {figures/yeast_1.csv};
        \addlegendentry{LR}
    \end{axis}
\end{tikzpicture}
\end{subfigure}

\begin{subfigure}{.5\textwidth}
\caption{Micro F1 score}
\label{fig:exp11}
\begin{tikzpicture}
    \begin{axis}[legend pos=north east,
        legend style={font=\fontsize{7}{5}\selectfont}, 
        xlabel={Degree of reduction},
        ylabel={Micro F1 score},
        width=7cm,
        xmax=1,
        ymax=0.8]
        \addplot table [x=k, y=LRWithPCA, col sep=semicolon] {figures/yeast_2.csv};
        \addlegendentry{LRWithPCA}
        \addplot table [x=k, y=CPLST, col sep=semicolon] {figures/yeast_2.csv};
        \addlegendentry{CPLST}
        \addplot table [x=k, y=LRWithRandomPCA, col sep=semicolon] {figures/yeast_2.csv};
        \addlegendentry{LRWithRandomPCA}
        \addplot table [x=k, y=OCCA, col sep=semicolon] {figures/yeast_2.csv};
        \addlegendentry{OCCA}
        \addplot table [x=k, y=LR, col sep=semicolon] {figures/yeast_2.csv};
        \addlegendentry{LR}
    \end{axis}
\end{tikzpicture}
\end{subfigure}
\begin{subfigure}{.5\textwidth}
\caption{Instance F1 score}
\label{fig:exp12}
\begin{tikzpicture}
    \begin{axis}[legend pos=north east,
        legend style={font=\fontsize{7}{5}\selectfont},
        xlabel={Degree of reduction},
        ylabel={Instance F1 score},
        width=7cm,
        xmax=1,
        ymax=0.8]
        \addplot table [x=k, y=LRWithPCA, col sep=semicolon] {figures/yeast_3.csv};
        \addlegendentry{LRWithPCA}
        \addplot table [x=k, y=CPLST, col sep=semicolon] {figures/yeast_3.csv};
        \addlegendentry{CPLST}
        \addplot table [x=k, y=LRWithRandomPCA, col sep=semicolon] {figures/yeast_3.csv};
        \addlegendentry{LRWithRandomPCA}
        \addplot table [x=k, y=OCCA, col sep=semicolon] {figures/yeast_3.csv};
        \addlegendentry{OCCA}
        \addplot table [x=k, y=LR, col sep=semicolon] {figures/yeast_3.csv};
        \addlegendentry{LR}
    \end{axis}
\end{tikzpicture}
\end{subfigure}

\end{figure}

\subsubsection{Bibtex data set}

\begin{table}[H]
\centering
\caption{Accuracy of methods for bibtex data set}
\label{tab:exp4}
\pgfplotstabletypeset [
    col sep=semicolon,
    columns/Method/.style={string type, column type={c|}},
    columns/Macro F1 score/.style={column type={c|}},
    columns/Micro F1 score/.style={column type={c|}},
    columns/Instance F1 score/.style={column type={c|}},
    columns/Hamming loss/.style={column type={c|}},
    every head row/.style={after row=\hline\hline}
]{figures/bibtex.csv}
\end{table}

\begin{figure}[H]
%\centering
\caption{Relation between reduction degree and accuracy for bibtex data set}
\label{fig:bibtex}

\begin{subfigure}{.5\textwidth}
\caption{Hamming loss}
\label{fig:exp13}
\begin{tikzpicture}
    \begin{axis}[legend pos=north east,
        legend style={font=\fontsize{7}{5}\selectfont}, 
        xlabel={Degree of reduction},
        ylabel={Hamming loss},
        width=7cm,
        xmax=1,
        ymax=0.02]
        \addplot table [x=k, y=LRWithPCA, col sep=semicolon] {figures/bibtex_0.csv};
        \addlegendentry{LRWithPCA}
        \addplot table [x=k, y=CPLST, col sep=semicolon] {figures/bibtex_0.csv};
        \addlegendentry{CPLST}
        \addplot table [x=k, y=LRWithRandomPCA, col sep=semicolon] {figures/bibtex_0.csv};
        \addlegendentry{LRWithRandomPCA}
        \addplot table [x=k, y=OCCA, col sep=semicolon] {figures/bibtex_0.csv};
        \addlegendentry{OCCA}
        \addplot table [x=k, y=LR, col sep=semicolon] {figures/bibtex_0.csv};
        \addlegendentry{LR}
    \end{axis}
\end{tikzpicture}
\end{subfigure}
\begin{subfigure}{.5\textwidth}
\caption{Macro F1 score}
\label{fig:exp14}
\begin{tikzpicture}
    \begin{axis}[legend pos=north east,
        legend style={font=\fontsize{7}{5}\selectfont}, 
        xlabel={Degree of reduction},
        ylabel={Macro F1 score},
        width=7cm,
        xmax=1,
        ymax=0.45]
        \addplot table [x=k, y=LRWithPCA, col sep=semicolon] {figures/bibtex_1.csv};
        \addlegendentry{LRWithPCA}
        \addplot table [x=k, y=CPLST, col sep=semicolon] {figures/bibtex_1.csv};
        \addlegendentry{CPLST}
        \addplot table [x=k, y=LRWithRandomPCA, col sep=semicolon] {figures/bibtex_1.csv};
        \addlegendentry{LRWithRandomPCA}
        \addplot table [x=k, y=OCCA, col sep=semicolon] {figures/bibtex_1.csv};
        \addlegendentry{OCCA}
        \addplot table [x=k, y=LR, col sep=semicolon] {figures/bibtex_1.csv};
        \addlegendentry{LR}
    \end{axis}
\end{tikzpicture}
\end{subfigure}

\begin{subfigure}{.5\textwidth}
\caption{Micro F1 score}
\label{fig:exp15}
\begin{tikzpicture}
    \begin{axis}[legend pos=north east,
        legend style={font=\fontsize{7}{5}\selectfont}, 
        xlabel={Degree of reduction},
        ylabel={Micro F1 score},
        width=7cm,
        xmax=1,
        ymax=0.75]
        \addplot table [x=k, y=LRWithPCA, col sep=semicolon] {figures/bibtex_2.csv};
        \addlegendentry{LRWithPCA}
        \addplot table [x=k, y=CPLST, col sep=semicolon] {figures/bibtex_2.csv};
        \addlegendentry{CPLST}
        \addplot table [x=k, y=LRWithRandomPCA, col sep=semicolon] {figures/bibtex_2.csv};
        \addlegendentry{LRWithRandomPCA}
        \addplot table [x=k, y=OCCA, col sep=semicolon] {figures/bibtex_2.csv};
        \addlegendentry{OCCA}
        \addplot table [x=k, y=LR, col sep=semicolon] {figures/bibtex_2.csv};
        \addlegendentry{LR}
    \end{axis}
\end{tikzpicture}
\end{subfigure}
\begin{subfigure}{.5\textwidth}
\caption{Instance F1 score}
\label{fig:exp16}
\begin{tikzpicture}
    \begin{axis}[legend pos=north east,
        legend style={font=\fontsize{7}{5}\selectfont},
        xlabel={Degree of reduction},
        ylabel={Instance F1 score},
        width=7cm,
        xmax=1,
        ymax=0.65]
        \addplot table [x=k, y=LRWithPCA, col sep=semicolon] {figures/bibtex_3.csv};
        \addlegendentry{LRWithPCA}
        \addplot table [x=k, y=CPLST, col sep=semicolon] {figures/bibtex_3.csv};
        \addlegendentry{CPLST}
        \addplot table [x=k, y=LRWithRandomPCA, col sep=semicolon] {figures/bibtex_3.csv};
        \addlegendentry{LRWithRandomPCA}
        \addplot table [x=k, y=OCCA, col sep=semicolon] {figures/bibtex_3.csv};
        \addlegendentry{OCCA}
        \addplot table [x=k, y=LR, col sep=semicolon] {figures/bibtex_3.csv};
        \addlegendentry{LR}
    \end{axis}
\end{tikzpicture}
\end{subfigure}


\end{figure}


\subsubsection{Corel16k data set}

\begin{table}[H]
\centering
\caption{Accuracy of methods for corel16k data set}
\label{tab:exp5}
\pgfplotstabletypeset [
    col sep=semicolon,
    columns/Method/.style={string type, column type={c|}},
    columns/Macro F1 score/.style={column type={c|}},
    columns/Micro F1 score/.style={column type={c|}},
    columns/Instance F1 score/.style={column type={c|}},
    columns/Hamming loss/.style={column type={c|}},
    every head row/.style={after row=\hline\hline}
]{figures/Corel16k001.csv}
\end{table}

\begin{figure}[H]
%\centering
\caption{Relation between reduction degree and accuracy for corel16k data set}
\label{fig:corel16k}

\begin{subfigure}{.5\textwidth}
\caption{Hamming loss}
\label{fig:exp17}
\begin{tikzpicture}
    \begin{axis}[legend pos=north east,
        legend style={font=\fontsize{7}{5}\selectfont}, 
        xlabel={Degree of reduction},
        ylabel={Hamming loss},
        width=7cm,
        xmax=1,
        ymax=0.02]
        \addplot table [x=k, y=LRWithPCA, col sep=semicolon] {figures/Corel16k001_0.csv};
        \addlegendentry{LRWithPCA}
        \addplot table [x=k, y=CPLST, col sep=semicolon] {figures/Corel16k001_0.csv};
        \addlegendentry{CPLST}
        \addplot table [x=k, y=LRWithRandomPCA, col sep=semicolon] {figures/Corel16k001_0.csv};
        \addlegendentry{LRWithRandomPCA}
        \addplot table [x=k, y=OCCA, col sep=semicolon] {figures/Corel16k001_0.csv};
        \addlegendentry{OCCA}
        \addplot table [x=k, y=LR, col sep=semicolon] {figures/Corel16k001_0.csv};
        \addlegendentry{LR}
    \end{axis}
\end{tikzpicture}
\end{subfigure}
\begin{subfigure}{.5\textwidth}
\caption{Macro F1 score}
\label{fig:exp18}
\begin{tikzpicture}
    \begin{axis}[legend pos=north east,
        legend style={font=\fontsize{7}{5}\selectfont}, 
        xlabel={Degree of reduction},
        ylabel={Macro F1 score},
        width=7cm,
        xmax=1,
        ymax=0.05]
        \addplot table [x=k, y=LRWithPCA, col sep=semicolon] {figures/Corel16k001_1.csv};
        \addlegendentry{LRWithPCA}
        \addplot table [x=k, y=CPLST, col sep=semicolon] {figures/Corel16k001_1.csv};
        \addlegendentry{CPLST}
        \addplot table [x=k, y=LRWithRandomPCA, col sep=semicolon] {figures/Corel16k001_1.csv};
        \addlegendentry{LRWithRandomPCA}
        \addplot table [x=k, y=OCCA, col sep=semicolon] {figures/Corel16k001_1.csv};
        \addlegendentry{OCCA}
        \addplot table [x=k, y=LR, col sep=semicolon] {figures/Corel16k001_1.csv};
        \addlegendentry{LR}
    \end{axis}
\end{tikzpicture}
\end{subfigure}

\begin{subfigure}{.5\textwidth}
\caption{Micro F1 score}
\label{fig:exp19}
\begin{tikzpicture}
    \begin{axis}[legend pos=north east,
        legend style={font=\fontsize{7}{5}\selectfont}, 
        xlabel={Degree of reduction},
        ylabel={Micro F1 score},
        ylabel style={yshift=0.4cm},
        width=7cm,
        xmax=1,
        ymax=0.15]
        \addplot table [x=k, y=LRWithPCA, col sep=semicolon] {figures/Corel16k001_2.csv};
        \addlegendentry{LRWithPCA}
        \addplot table [x=k, y=CPLST, col sep=semicolon] {figures/Corel16k001_2.csv};
        \addlegendentry{CPLST}
        \addplot table [x=k, y=LRWithRandomPCA, col sep=semicolon] {figures/Corel16k001_2.csv};
        \addlegendentry{LRWithRandomPCA}
        \addplot table [x=k, y=OCCA, col sep=semicolon] {figures/Corel16k001_2.csv};
        \addlegendentry{OCCA}
        \addplot table [x=k, y=LR, col sep=semicolon] {figures/Corel16k001_2.csv};
        \addlegendentry{LR}
    \end{axis}
\end{tikzpicture}
\end{subfigure}
\begin{subfigure}{.5\textwidth}
\caption{Instance F1 score}
\label{fig:exp20}
\begin{tikzpicture}
    \begin{axis}[legend pos=north east,
        legend style={font=\fontsize{7}{5}\selectfont},
        xlabel={Degree of reduction},
        ylabel={Instance F1 score},
        ylabel style={yshift=0.4cm},
        width=7cm,
        xmax=1,
        ymax=0.13]
        \addplot table [x=k, y=LRWithPCA, col sep=semicolon] {figures/Corel16k001_3.csv};
        \addlegendentry{LRWithPCA}
        \addplot table [x=k, y=CPLST, col sep=semicolon] {figures/Corel16k001_3.csv};
        \addlegendentry{CPLST}
        \addplot table [x=k, y=LRWithRandomPCA, col sep=semicolon] {figures/Corel16k001_3.csv};
        \addlegendentry{LRWithRandomPCA}
        \addplot table [x=k, y=OCCA, col sep=semicolon] {figures/Corel16k001_3.csv};
        \addlegendentry{OCCA}
        \addplot table [x=k, y=LR, col sep=semicolon] {figures/Corel16k001_3.csv};
        \addlegendentry{LR}
    \end{axis}
\end{tikzpicture}
\end{subfigure}


\end{figure}

\subsubsection{EUR-Lex data set}

\begin{table}[H]
\centering
\caption{Accuracy of methods for EUR-Lex data set}
\label{tab:exp6}
\pgfplotstabletypeset [
    col sep=semicolon,
    columns/Method/.style={string type, column type={c|}},
    columns/Macro F1 score/.style={column type={c|}},
    columns/Micro F1 score/.style={column type={c|}},
    columns/Instance F1 score/.style={column type={c|}},
    columns/Hamming loss/.style={column type={c|}},
    every head row/.style={after row=\hline\hline}
]{figures/eurlex.csv}
\end{table}

\begin{figure}[H]
%\centering
\caption{Relation between reduction degree and accuracy for EUR-Lex data set}
\label{fig:eurlex}

\begin{subfigure}{.5\textwidth}
\caption{Hamming loss}
\label{fig:exp21}
\begin{tikzpicture}
    \begin{axis}[legend pos=north east,
        legend style={font=\fontsize{7}{5}\selectfont}, 
        xlabel={Degree of reduction},
        ylabel={Hamming loss},
        width=7cm,
        xmax=1,
        ymax=0.05]
        \addplot table [x=k, y=LRWithPCA, col sep=semicolon] {figures/eurlex_0.csv};
        \addlegendentry{LRWithPCA}
        \addplot table [x=k, y=CPLST, col sep=semicolon] {figures/eurlex_0.csv};
        \addlegendentry{CPLST}
        \addplot table [x=k, y=LRWithRandomPCA, col sep=semicolon] {figures/eurlex_0.csv};
        \addlegendentry{LRWithRandomPCA}
        \addplot table [x=k, y=OCCA, col sep=semicolon] {figures/eurlex_0.csv};
        \addlegendentry{OCCA}
        \addplot table [x=k, y=LR, col sep=semicolon] {figures/eurlex_0.csv};
        \addlegendentry{LR}
    \end{axis}
\end{tikzpicture}
\end{subfigure}
\begin{subfigure}{.5\textwidth}
\caption{Macro F1 score}
\label{fig:exp22}
\begin{tikzpicture}
    \begin{axis}[legend pos=north east,
        legend style={font=\fontsize{7}{5}\selectfont}, 
        xlabel={Degree of reduction},
        ylabel={Macro F1 score},
        ylabel style={yshift=0.4cm},
        width=7cm,
        xmax=1,
        ymax=0.15]
        \addplot table [x=k, y=LRWithPCA, col sep=semicolon] {figures/eurlex_1.csv};
        \addlegendentry{LRWithPCA}
        \addplot table [x=k, y=CPLST, col sep=semicolon] {figures/eurlex_1.csv};
        \addlegendentry{CPLST}
        \addplot table [x=k, y=LRWithRandomPCA, col sep=semicolon] {figures/eurlex_1.csv};
        \addlegendentry{LRWithRandomPCA}
        \addplot table [x=k, y=OCCA, col sep=semicolon] {figures/eurlex_1.csv};
        \addlegendentry{OCCA}
        \addplot table [x=k, y=LR, col sep=semicolon] {figures/eurlex_1.csv};
        \addlegendentry{LR}
    \end{axis}
\end{tikzpicture}
\end{subfigure}

\begin{subfigure}{.5\textwidth}
\caption{Micro F1 score}
\label{fig:exp23}
\begin{tikzpicture}
    \begin{axis}[legend pos=north east,
        legend style={font=\fontsize{7}{5}\selectfont}, 
        xlabel={Degree of reduction},
        ylabel={Micro F1 score},
        width=7cm,
        xmax=1,
        ymax=0.4]
        \addplot table [x=k, y=LRWithPCA, col sep=semicolon] {figures/eurlex_2.csv};
        \addlegendentry{LRWithPCA}
        \addplot table [x=k, y=CPLST, col sep=semicolon] {figures/eurlex_2.csv};
        \addlegendentry{CPLST}
        \addplot table [x=k, y=LRWithRandomPCA, col sep=semicolon] {figures/eurlex_2.csv};
        \addlegendentry{LRWithRandomPCA}
        \addplot table [x=k, y=OCCA, col sep=semicolon] {figures/eurlex_2.csv};
        \addlegendentry{OCCA}
        \addplot table [x=k, y=LR, col sep=semicolon] {figures/eurlex_2.csv};
        \addlegendentry{LR}
    \end{axis}
\end{tikzpicture}
\end{subfigure}
\begin{subfigure}{.5\textwidth}
\caption{Instance F1 score}
\label{fig:exp24}
\begin{tikzpicture}
    \begin{axis}[legend pos=north east,
        legend style={font=\fontsize{7}{5}\selectfont},
        xlabel={Degree of reduction},
        ylabel={Instance F1 score},
        width=7cm,
        xmax=1,
        ymax=0.65]
        \addplot table [x=k, y=LRWithPCA, col sep=semicolon] {figures/eurlex_3.csv};
        \addlegendentry{LRWithPCA}
        \addplot table [x=k, y=CPLST, col sep=semicolon] {figures/eurlex_3.csv};
        \addlegendentry{CPLST}
        \addplot table [x=k, y=LRWithRandomPCA, col sep=semicolon] {figures/eurlex_3.csv};
        \addlegendentry{LRWithRandomPCA}
        \addplot table [x=k, y=OCCA, col sep=semicolon] {figures/eurlex_3.csv};
        \addlegendentry{OCCA}
    \end{axis}
\end{tikzpicture}
\end{subfigure}


\end{figure}

\subsubsection{Bookmarks data set}

\begin{table}[H]
\centering
\caption{Accuracy of methods for bookmarks data set}
\label{tab:exp7}
\pgfplotstabletypeset [
    col sep=semicolon,
    columns/Method/.style={string type, column type={c|}},
    columns/Macro F1 score/.style={column type={c|}},
    columns/Micro F1 score/.style={column type={c|}},
    columns/Instance F1 score/.style={column type={c|}},
    columns/Hamming loss/.style={column type={c|}},
    every head row/.style={after row=\hline\hline}
]{figures/bookmarks.csv}
\end{table}

\begin{figure}[H]
%\centering
\caption{Relation between reduction degree and accuracy for bookmarks data set}
\label{fig:bookmarks}

\begin{subfigure}{.5\textwidth}
\caption{Hamming loss}
\label{fig:exp25}
\begin{tikzpicture}
    \begin{axis}[legend pos=north east,
        legend style={font=\fontsize{7}{5}\selectfont}, 
        xlabel={Degree of reduction},
        ylabel={Hamming loss},
        width=7cm,
        xmax=1,
        ymax=0.01]
        \addplot table [x=k, y=LRWithPCA, col sep=semicolon] {figures/bookmarks_0.csv};
        \addlegendentry{LRWithPCA}
        \addplot table [x=k, y=CPLST, col sep=semicolon] {figures/bookmarks_0.csv};
        \addlegendentry{CPLST}
        \addplot table [x=k, y=LRWithRandomPCA, col sep=semicolon] {figures/bookmarks_0.csv};
        \addlegendentry{LRWithRandomPCA}
        \addplot table [x=k, y=OCCA, col sep=semicolon] {figures/bookmarks_0.csv};
        \addlegendentry{OCCA}
        \addplot table [x=k, y=LR, col sep=semicolon] {figures/bookmarks_0.csv};
        \addlegendentry{LR}
    \end{axis}
\end{tikzpicture}
\end{subfigure}
\begin{subfigure}{.5\textwidth}
\caption{Macro F1 score}
\label{fig:exp26}
\begin{tikzpicture}
    \begin{axis}[legend pos=north east,
        legend style={font=\fontsize{7}{5}\selectfont}, 
        xlabel={Degree of reduction},
        ylabel={Macro F1 score},
        ylabel style={yshift=0.4cm},
        width=7cm,
        xmax=1,
        ymax=0.13]
        \addplot table [x=k, y=LRWithPCA, col sep=semicolon] {figures/bookmarks_1.csv};
        \addlegendentry{LRWithPCA}
        \addplot table [x=k, y=CPLST, col sep=semicolon] {figures/bookmarks_1.csv};
        \addlegendentry{CPLST}
        \addplot table [x=k, y=LRWithRandomPCA, col sep=semicolon] {figures/bookmarks_1.csv};
        \addlegendentry{LRWithRandomPCA}
        \addplot table [x=k, y=OCCA, col sep=semicolon] {figures/bookmarks_1.csv};
        \addlegendentry{OCCA}
        \addplot table [x=k, y=LR, col sep=semicolon] {figures/bookmarks_1.csv};
        \addlegendentry{LR}
    \end{axis}
\end{tikzpicture}
\end{subfigure}

\begin{subfigure}{.5\textwidth}
\caption{Micro F1 score}
\label{fig:exp27}
\begin{tikzpicture}
    \begin{axis}[legend pos=north east,
        legend style={font=\fontsize{7}{5}\selectfont}, 
        xlabel={Degree of reduction},
        ylabel={Micro F1 score},
        width=7cm,
        xmax=1,
        ymax=0.3]
        \addplot table [x=k, y=LRWithPCA, col sep=semicolon] {figures/bookmarks_2.csv};
        \addlegendentry{LRWithPCA}
        \addplot table [x=k, y=CPLST, col sep=semicolon] {figures/bookmarks_2.csv};
        \addlegendentry{CPLST}
        \addplot table [x=k, y=LRWithRandomPCA, col sep=semicolon] {figures/bookmarks_2.csv};
        \addlegendentry{LRWithRandomPCA}
        \addplot table [x=k, y=OCCA, col sep=semicolon] {figures/bookmarks_2.csv};
        \addlegendentry{OCCA}
        \addplot table [x=k, y=LR, col sep=semicolon] {figures/bookmarks_2.csv};
        \addlegendentry{LR}
    \end{axis}
\end{tikzpicture}
\end{subfigure}
\begin{subfigure}{.5\textwidth}
\caption{Instance F1 score}
\label{fig:exp28}
\begin{tikzpicture}
    \begin{axis}[legend pos=north east,
        legend style={font=\fontsize{7}{5}\selectfont},
        xlabel={Degree of reduction},
        ylabel={Instance F1 score},
        width=7cm,
        xmax=1,
        ymax=0.25]
        \addplot table [x=k, y=LRWithPCA, col sep=semicolon] {figures/bookmarks_3.csv};
        \addlegendentry{LRWithPCA}
        \addplot table [x=k, y=CPLST, col sep=semicolon] {figures/bookmarks_3.csv};
        \addlegendentry{CPLST}
        \addplot table [x=k, y=LRWithRandomPCA, col sep=semicolon] {figures/bookmarks_3.csv};
        \addlegendentry{LRWithRandomPCA}
        \addplot table [x=k, y=OCCA, col sep=semicolon] {figures/bookmarks_3.csv};
        \addlegendentry{OCCA}
        \addplot table [x=k, y=LR, col sep=semicolon] {figures/bookmarks_3.csv};
        \addlegendentry{LR}
    \end{axis}
\end{tikzpicture}
\end{subfigure}


\end{figure}

\newpage
\subsection{Comparison of methods efficiency}

The efficiency experiment aimed at measuring time of training classifier by particular methods and was performed on bookmarks data set with a changeable number of instances. The computation was run in single- and multi-threaded environment, in order to show how multithreading affects the efficiency. The test involved the following algorithms:
\begin{itemize}
    \item CPLST,
    \item LR,
    \item LRWithPCA,
    \item LRWithRPCA(L) which is linear regression with random PCA compression on a label space,
    \item LRWithRPCA(F) which is linear regression with random PCA compression on a feature space,
    \item OCCA.
\end{itemize}
Due to meaningful differences (even about an order of magnitude) between particular algorithms, the results are shown in three figures: \Crefrange{fig:exp29}{fig:exp31}.

\newpage
\subsubsection{Single-threaded performance}

\begin{figure}[H]
\centering
\caption{Efficiency of single-threaded LR and RPCA classifiers}
\label{fig:exp29}
\begin{tikzpicture}[scale=0.93]
    \begin{axis}[legend pos=north east,
        xlabel={number of instances},
        ylabel={time [s]},
        ymax={60}]
        \addplot table [x=instance, y=time, col sep=semicolon] {figures/LR1.csv};
        \addlegendentry{LR}
        \addplot table [x=instance, y=time, col sep=semicolon] {figures/LRWithRandomPCA1.csv};
        \addlegendentry{LRWithRPCA(L)}
        \addplot table [x=instance, y=time, col sep=semicolon] {figures/LRWithRandomPCA_d1.csv};
        \addlegendentry{LRWithRPCA(F)}

    \end{axis}
\end{tikzpicture}
\end{figure}

\begin{figure}[H]
\centering
\caption{Efficiency of single-threaded OCCA and CPLST}
\label{fig:exp30}
\begin{tikzpicture}[scale=0.93]
    \begin{axis}[legend pos=north east,
        xlabel={number of instances},
        ylabel={time [s]},
        ymax={1500}]
        \addplot table [x=instance, y=time, col sep=semicolon] {figures/CPLST1.csv};
        \addlegendentry{CPLST}
        \addplot table [x=instance, y=time, col sep=semicolon] {figures/OCCA1.csv};
        \addlegendentry{OCCA}

    \end{axis}
\end{tikzpicture}
\end{figure}

\begin{figure}[H]
\centering
\caption{Efficiency of single-threaded LRWithPCA}
\label{fig:exp31}
\begin{tikzpicture}[scale=0.93]
    \begin{axis}[legend pos=north east,
        xlabel={number of instances},
        ylabel={time [s]},
        ymax={5000}]
        \addplot table [x=instance, y=time, col sep=semicolon] {figures/LRWithPCA1.csv};
        \addlegendentry{LRWithPCA}

    \end{axis}
\end{tikzpicture}
\end{figure}

\subsubsection{Multi-threaded performance}

\begin{figure}[H]
\centering
\caption{Efficiency of multi-threaded LR and RPCA classifiers}
\label{fig:exp32}
\begin{tikzpicture}[scale=0.93]
    \begin{axis}[legend pos=north east,
        xlabel={number of instances},
        ylabel={time [s]},
        ymax=16]
        \addplot table [x=instance, y=time, col sep=semicolon] {figures/LR.csv};
        \addlegendentry{LR}
        \addplot table [x=instance, y=time, col sep=semicolon] {figures/LRWithRandomPCA.csv};
        \addlegendentry{LRWithRPCA(L)}
        \addplot table [x=instance, y=time, col sep=semicolon] {figures/LRWithRandomPCA_d.csv};
        \addlegendentry{LRWithRPCA(F)}

    \end{axis}
\end{tikzpicture}
\end{figure}

\begin{figure}[H]
\centering
\caption{Efficiency of multi-threaded OCCA and CPLST}
\label{fig:exp33}
\begin{tikzpicture}[scale=0.93]
    \begin{axis}[legend pos=north east,
        xlabel={number of instances},
        ylabel={time [s]},
        ymax=300]
        \addplot table [x=instance, y=time, col sep=semicolon] {figures/CPLST.csv};
        \addlegendentry{CPLST}
        \addplot table [x=instance, y=time, col sep=semicolon] {figures/OCCA.csv};
        \addlegendentry{OCCA}

    \end{axis}
\end{tikzpicture}
\end{figure}

\begin{figure}[H]
\centering
\caption{Efficiency of multi-threaded LRWithPCA}
\label{fig:exp34}
\begin{tikzpicture}[scale=0.93]
    \begin{axis}[legend pos=north east,
        xlabel={number of instances},
        ylabel={time [s]},
        ymax=2500]
        \addplot table [x=instance, y=time, col sep=semicolon] {figures/LRWithPCA.csv};
        \addlegendentry{LRWithPCA}
    \end{axis}
\end{tikzpicture}
\end{figure}

\chapter{Conclusion}

\section{Discussing quality experiment results}

The quality of classification for all the methods is strongly dependent on the training sets and the metric which is taken into account. The most 'difficult' training set is \textit{corel16k}, even though it is not the biggest set - its number of features, number of labels and number of instances as well are lower than in case of \textit{bookmarks} or \textit{EUR-Lex} data sets. Besides \textit{Hamming loss} metric, which is satisfying, the rest of them is levelled of $10^{-2}$ (\Cref{tab:exp5} and \Cref{fig:corel16k}) what is a very low score. As we see, a nature and a domain of data set also affect the results. On the other hand, a domain is not a dominant factor. \textit{Scene} data set, as well as \textit{corel16k}, is connected with images, but the results are far better for the first one (\Cref{tab:exp3}). It is worth mentioning that high scores were also reached for \textit{yeast} set - its domain is 'biology'.

The most effective approach, when it comes about quality and regardless of the metrics, is \textit{LR}. This is not certainly suprising - the rest of the methods use compression which causes a fall of the quality metrics. Among the algorithms which use compression, the best results were given by \textit{CPLST}. This is also rather predictible, because the reduction of a label space is feature awared unlike \textit{PCA} compression. Nevertheless, we should remember that all the approaches use a linear regressor to create a classifier. Thus behaviours of the algorithms are rather similar for particular datasets. For instance the characteristic features of all the methods are a low score of \textit{Macro-average F1} and a satisfying score of \textit{Hamming loss} for the majority of the datasets used in the experiment. 

\Crefrange{fig:enron}{fig:bookmarks} show the relation between a value of a reduction degree and a value of specific metrics. As we see the lower reduction degree we use, the worse scores we get in almost all cases. An interesting anomaly appears in \textit{EUR-Lex} dataset. In this particular case \textit{Hamming loss} increases regularly with a growth of a reduction degree value. The situation is similar when it comes about \textit{Micro-average F1} (it should grow but it decreases). The domain of \textit{EUR-Lex} is 'text' - this set is connected with \textit{TF-IDF} representation of documents. It is possible that most of the features are useless and make a noise in the data - it could explain the anomaly.  

\section{Discussing efficency experiment results}

All the methods, which took part in the experiment, perform faster in the multi-threaded environment. The difference is noticeable - for some algorithms a computation is even four times faster. It is certainly the result of efficient matrix operations (matrix multiplication etc.) which are suitable for parallelization. It is also obvious that time-consumption grows with a number of instances.

The best approach, when it comes about time-consumption, is the linear regression (\Cref{fig:exp29} and \Cref{fig:exp32}). Although the other methods create a classifier from reduced data, the time cost of such the compression is too high. However, the results reached by \textit{LRWithRPCA} are not much worse than in case of \textit{LR}. Unlike the rest of the approaches based on preliminary compression, \textit{LRWithRPCA} doses not use \textit{SVD} decomposition. This fact explains why this method is much faster - it is worth reminding that \textit{SVD} has the complexity of $O(m^2n+n^3)$. 

\Cref{fig:exp31,fig:exp34} show that the slowest approach is \textit{LRWithPCA}. We should remember that only a label space was compressed in the experiment - compressing a feature space is too time-consuming in case of \textit{LRWithPCA}. As we see, this implementation is useless for massive data. 

\textit{CPLST} and \textit{OCCA} (\Cref{fig:exp30,fig:exp33}) have almost the same time complexity. It is certainly not suprising - both the methods process matrices of the same size in a very similar way.

\section{Future research implications}

The efficency experiment has shown that parallelizing of computation has huge influence on time-consumption. Therefore it is sensible to implement the algorithms for environments which use GPU to process. This will probably improve the efficiency for all the methods. It is also worth checking if an involvement of greater amount of cores (threads) in processing decreases time-consumption.  

Among the tested algorithms there are two approaches which can be rejected: \textit{LRWithPCA} and \textit{OCCA}. The first one is definitely too slow and is also not satisfying when it comes about the quality. In fact, its random version gives similar quality results (sometimes even better). The second one is exactly as fast as \text{CPLST} algorithm, but reached worse scores in the quality experiment. 

The most suprising aspect of the research is the result achieved by \textit{LR} algorithm in both the experiments - especially in the second one. The transformations of feature and labels spaces have aimed at decreasing input matrices for the linear regressor. Unfortunately these transformations are too time-consuming towards building the regressor. However, the time achieved by \textit{LRWithRPCA} is not much worse. Moreover it is possible to manipulate the specific parameters of this method what can help obtain better results. All in all there is a lot to investigate in this particular approach.   

Even though \textit{CPLST} is not as efficient as \textit{LR} and \textit{LRWithRPCA}, this approach should also be studied - more generally, methods inspired by \textit{CCA} should be studied. Let us remind that the algorithm reached better results than the algorithms which use \textit{PCA} in the quality experiment. In this specific implementation of \textit{CPLST}, a linear regressor is used to build a final classifier - it is sensible to check different solutions, for example a logistic regressor etc. It is also sensible to check how some different regressor would cooperate with \textit{RandomPCA} algorithm.

\section{Summery}

The scope of the thesis has involved fast implementations of multi-label classification algorithms based on preliminary transformations of feature and label spaces. This goal has been achieved - the efficency experiment showed that the methods 'learn' fast, even from massive data. However, as it has been mentioned in the previous section, there are still many approaches to test.  

It is significant to remeber that such a good efficency is the result of the calculation parallelizing which were possible by the fast linear algebra libraries. It is also worth noting that the best scores were reached by the standard linear regression algorithm which is rather a simple approach. In fact, it is sensible to consider simpler models which can be effectively parallelized and improve them instead of developing complicated methods. 

The implementations of the algorithms, discussed in this paper, was written in \textit{C++11} programming language and became a part of the library called \textit{MLCPACK}. The library was organised in a flexible way (\textit{strategy} design pattern) and provides the interface which can be used to add new algorithms. Besides the algorithms, the library contains additional tools, such as \textit{ARFF} parser or quality evaluators. It is woth mentioning that \textit{MLCPACK} is prepared for performing in a multi-threaded environment.


% All appendices and extra material, if you have any.
\cleardoublepage
\begin{appendices}

\chapter{Proofs and facts}

\section{Least squares for multiple regression}
\label{app:least}

Given a design matrix $\boldsymbol{X} \in \mathbb{R}^{m \times (n+1)}$ and a vector of real values $\boldsymbol{y} \in \mathbb{R}^{n}$, we want to find $\boldsymbol{\beta}$, such that:
\begin{equation}\label{eq:least0}
    \boldsymbol{\hat{\beta}} = \argmin_{\boldsymbol{\beta} \in \mathbb{R}^{n+1}}(\boldsymbol{y}-\boldsymbol{X}\boldsymbol{\beta})^T(\boldsymbol{y}-\boldsymbol{X}\boldsymbol{\beta})
\end{equation}
Let us notice that we can treat $(\boldsymbol{y}-\boldsymbol{X}\boldsymbol{\beta})^T(\boldsymbol{y}-\boldsymbol{X}\boldsymbol{\beta})$ as the function of $\boldsymbol{\beta}$. So, we can solve \Cref{eq:least0} by finding the global minimum of this function. This, of course, can be achieved by calculating the derivative in respect to $\boldsymbol{\beta}$:
\begin{equation}\label{eq:least1}
    \nabla_{\boldsymbol{\beta}} \frac{1}{2}(\boldsymbol{y}-\boldsymbol{X}\boldsymbol{\beta})^T(\boldsymbol{y}-\boldsymbol{X}\boldsymbol{\beta}) = 0
\end{equation}
We will solve the above equation by using the relations between derivatives with matrices and the trace of a matrix \citep{LRV}. First of all, let us remind that for a square matrix $\boldsymbol{A} \in \mathbb{R}^{n \times n}$ its trace is defined as: $tr(\boldsymbol{A})= \sum\limits_{i=1}^{n}a_{ii}$. We also note a few facts about the trace:

\begin{enumerate}
    \item $tr(\boldsymbol{A}+\boldsymbol{B}) = tr(\boldsymbol{A})+tr(\boldsymbol{B})$
    \item $tr(\boldsymbol{AB}) = tr(\boldsymbol{BA})$
    \item $tr(\boldsymbol{ABC}) = tr(\boldsymbol{CAB}) = tr(\boldsymbol{BCA})$
    \item $\nabla_{\boldsymbol{A}}tr(\boldsymbol{AB}) = \boldsymbol{B}^T$
    \item $tr(\boldsymbol{A})=tr(\boldsymbol{A}^T)$
    \item if $a \in \mathbb{R}$ then $tr(a)=a$
    \item $\nabla_{\boldsymbol{A}}tr(\boldsymbol{ABA}^T\boldsymbol{C}) = \boldsymbol{CAB}+\boldsymbol{C}^T\boldsymbol{AB}^T$
\end{enumerate}
Calculating a dot product in \Cref{eq:least1} and using the property (6.) of a trace operator, we obtain:
\begin{equation}\label{eq:least2}
    \frac{1}{2} \nabla_{\boldsymbol{\beta}} tr(\boldsymbol{\beta}^T\boldsymbol{X}^T\boldsymbol{X}\boldsymbol{\beta}-\boldsymbol{\beta}^T\boldsymbol{X}^T\boldsymbol{y}-\boldsymbol{y}^T\boldsymbol{X}\boldsymbol{\beta}+\boldsymbol{y}^T\boldsymbol{y}) = 0
\end{equation}
Next, using the properties (1.), (3.) and again (6.), \Cref{eq:least2} can be expressed as:
\begin{equation}\label{eq:least3}
    \frac{1}{2}[\nabla_{\boldsymbol{\beta}}tr(\boldsymbol{\beta}\boldsymbol{\beta}^T\boldsymbol{X}^T\boldsymbol{X})-\nabla_{\boldsymbol{\beta}}tr(\boldsymbol{y}^T\boldsymbol{X}\boldsymbol{\beta})-\nabla_{\boldsymbol{\beta}}tr(\boldsymbol{y}^T\boldsymbol{X}\boldsymbol{\beta})] = 0
\end{equation}
The derivative of $\boldsymbol{y}^T\boldsymbol{y}$ is $0$, because it does not depend on $\boldsymbol{\beta}$. Thus, it has been removed. Let us also notice that:
\begin{equation}\label{eq:least4}
    \nabla_{\boldsymbol{\beta}}tr(\boldsymbol{\beta}\boldsymbol{\beta}^T\boldsymbol{X}^T\boldsymbol{X}) = \nabla_{\boldsymbol{\beta}}tr(\boldsymbol{\beta}\boldsymbol{I}\boldsymbol{\beta}^T\boldsymbol{X}^T\boldsymbol{X}) = \boldsymbol{X}^T\boldsymbol{X}\boldsymbol{\beta}I + \boldsymbol{X}^T\boldsymbol{X}\boldsymbol{\beta}\boldsymbol{I} = \boldsymbol{X}^T\boldsymbol{X}\boldsymbol{\beta}+\boldsymbol{X}^T\boldsymbol{X}\boldsymbol{\beta}
\end{equation}
because of the property (7.) of a trace operator and:
\begin{equation}\label{eq:least5}
    \nabla_{\boldsymbol{\beta}}tr(\boldsymbol{y}^T\boldsymbol{X}\boldsymbol{\beta}) = \boldsymbol{X}^T\boldsymbol{y}
\end{equation}
because of the property (4.).
Now we can simply use \Cref{eq:least4} and \Cref{eq:least5} in \Cref{eq:least3} and calculate the desired derivative:
\begin{equation}\label{eq:least6}
    \frac{1}{2}[\boldsymbol{X}^T\boldsymbol{X}\boldsymbol{\beta}+\boldsymbol{X}^T\boldsymbol{X}\boldsymbol{\beta}-\boldsymbol{X}^T\boldsymbol{y}-\boldsymbol{X}^T\boldsymbol{y}] = 0
\end{equation}
Finally, we obtain $\boldsymbol{\beta}$ from \Cref{eq:least6}:
\begin{equation}
    \boldsymbol{\beta}=(\boldsymbol{X}^T\boldsymbol{X})^{-1}\boldsymbol{X}^T\boldsymbol{y}
\end{equation}

\newpage
\section{Eigenvalues and eigenvectors of matrices}
\label{app:eigen}

Let a matrix $\boldsymbol{T} \in \mathbb{R}^{m \times n}$ be a linear transformation matrix. Such matrix can certainly express primary linear transformations, for instance rotation, scaling, transposition, etc. If there are a column vector $\boldsymbol{v}$ and a scalar $\lambda$, such that:
\begin{equation}\label{eq:eigen1}
    \boldsymbol{T}\boldsymbol{v}=\lambda{\boldsymbol{v}}
\end{equation}
then we call them an eigenvector and suitably an eigenvalue of a matrix $\boldsymbol{T}$. Intuitively an eigenvector $\boldsymbol{v}$ of a matrix $\boldsymbol{T}$ is transformed by $\boldsymbol{T}$ into itself. An eigenvalue can be then interpreted as a scale factor. Formally eigenvectors and eigenvalues characterize an endomorphism of a particular linear space.  

The equation \ref{eq:eigen1} can be certainly stated equivalently as:

\begin{equation}\label{eq:eigen2}
    (\boldsymbol{T}-\lambda{\boldsymbol{I}})\boldsymbol{v}=0
\end{equation}
where $\boldsymbol{I}$ is the identity matrix. Let us notice that \Cref{eq:eigen2} has a solution if and only if the determinant of a matrix $(\boldsymbol{T}-\lambda{\boldsymbol{I}})$ is equal to zero. As a result, we can find eigenvalues by solving the following equation:
\begin{equation}\label{eq:eigen3}
    det(\boldsymbol{T}-\lambda{\boldsymbol{I}}) = (\lambda_1-\lambda)(\lambda_2-\lambda)\cdots(\lambda_n-\lambda) = 0
\end{equation}
\Cref{eq:eigen3} is called the characteristic equation, while its left-hand side is called the characteristic polynomial \citep{Banerjee}. In practice this method is not used to calculate eigenvectors and eigenvalues. The most popular algorithm is \textit{Power iteration} which is also called \textit{Von Mises iteration}.


\newpage
\section{Singular value decomposition}
\label{app:svd}

SVD decomposition is a particular factorization of a real or a complex matrix. Formally, each arbitrary matrix $\boldsymbol{X} \in \mathbb{R}^{n \times p}$ can be written as \citep{Jolliffe}:
\begin{equation}
    \boldsymbol{X}=\boldsymbol{U}\boldsymbol{\Sigma}{\boldsymbol{V}^T}
\end{equation}
where:
\begin{enumerate}
    \item $\boldsymbol{U}$ and $\boldsymbol{V}$ are $(n \times r)$, $(p \times r)$ matrices which have orthonormal columns. It means that $\boldsymbol{U}\boldsymbol{U}^T=\boldsymbol{I}$ and $\boldsymbol{V}\boldsymbol{V}^T=\boldsymbol{I}$, where $\boldsymbol{I}$ is the identity matrix,
    \item $\boldsymbol{\Sigma}$ is an $(r \times r)$ matrix,
    \item $r$ is a rank of $\boldsymbol{X}$.
\end{enumerate}
If $\boldsymbol{X}$ is a symmetric matrix, its decomposition is equivalent to:
\begin{equation}
    \boldsymbol{X}=\boldsymbol{V}\boldsymbol{\Sigma}{\boldsymbol{V}^T}
\end{equation}
The factorization with such properties can be computed using the following observations:
\begin{itemize}
    \item The columns of $\boldsymbol{U}$ matrix are eigenvectors of $\boldsymbol{X}\boldsymbol{X}^T$.
    \item The columns of $\boldsymbol{V}$ matrix are eigenvectors of $\boldsymbol{X}^T\boldsymbol{X}$.
    \item The values found on the diagonal entries of $\boldsymbol{\Sigma}$ are square roots of non-zero eigenvalues of both $\boldsymbol{X}\boldsymbol{X}^T$ and $\boldsymbol{X}^T\boldsymbol{X}$.
\end{itemize}

Besides a few interesting mathematical properties, the SVD decomposition has also an intuitive interpretation. If we assume that $\boldsymbol{X} \in \mathbb{R}^{n \times n}$ is a square matrix of a linear transformation with the positive determinant, then $\boldsymbol{\Sigma}$ can be thought as a scaling matrix while $\boldsymbol{V}^T$ and $\boldsymbol{U}$ can be viewed as rotation matrices. Therefore the expression $\boldsymbol{U}\boldsymbol{\Sigma}{\boldsymbol{V}^T}$ can be regarded as a composition of three geometrical transformations: two rotations and scaling.\footnote{\bibentry{SVD}}

The standard SVD algorithm has complexity of $O(np^2+p^3)$ and consists off two steps. In the first step, an origin matrix is transformed into a bidiagonal matrix. It is usually made by \textit{Householder reflection} algorithm which is $O(np^2+p^3)$ in the worst case. The second step is connected with proceeding SVD on a reduced matrix obtained in the first step. This part of the algorithm costs $O(p^2)$ and is computed by \textit{QR} decomposition \citep{Banerjee}. 

\newpage
\section{Pseudoinverse of matrix}\label{app:pseudo}

In this section we will define the pseudoinverse of a matrix $\boldsymbol{X}$ which is used if $\boldsymbol{X}$ is not invertible. Let us consider two cases: 
\begin{itemize}
    \item $\boldsymbol{X}$ is a diagonal square matrix,
    \item $\boldsymbol{X}$ is an arbitrary square matrix.
\end{itemize}
Let us remind that in order to compute inverse of a diagonal matrix, we have to inverse its diagonal elements. The pseudoinverse of such matrix is very similar -- we simply inverse elements which are not equal to $0$. The second case is more complex. First of all let us decompose $\boldsymbol{X}$ by SVD (\Cref{app:svd}):
\begin{equation}
    \boldsymbol{X}=\boldsymbol{U}\boldsymbol{\Sigma}{\boldsymbol{V}^T}
\end{equation}
Now, using the fact that $(\boldsymbol{A}\boldsymbol{B})^{-1}=\boldsymbol{B}^{-1}\boldsymbol{A}^{-1}$, we can express the inverse of $\boldsymbol{X}$ in the following way:
\begin{equation}\label{eq:invert}
    \boldsymbol{X}^{-1}=(\boldsymbol{U}\boldsymbol{\Sigma}{\boldsymbol{V}^T})^{-1}=(\boldsymbol{V}^T)^{-1}{\boldsymbol{\Sigma}}^{-1}\boldsymbol{U}^{-1}
\end{equation}
$\boldsymbol{V}$ and $\boldsymbol{U}$ matrices are orthogonal so we can simplify \Cref{eq:invert}:
\begin{equation}
    \boldsymbol{X}^{-1}=\boldsymbol{V}{\boldsymbol{\Sigma}}^{-1}\boldsymbol{U}^T
\end{equation}
As we see, the problem is with the ivertibility of $\boldsymbol{\Sigma}$ matrix. However, $\boldsymbol{\Sigma}$ is diagonal and we have defined the pseudoinverse for such a matrix. Finally, we obtain the following definition:
\begin{equation}
    \boldsymbol{X}^P=\boldsymbol{V}\boldsymbol{\Sigma}^{P}{\boldsymbol{U}^T}
\end{equation}

\newpage
\section{Invertibility of matrix in shrinkage regression}
\label{app:reg}

In this section we will show that it is not necessary to use pseudoinverse in shrinkage regression algorithm. Let us remind that the solution of shrinkage regression problem is given by the following formula:

\begin{equation}\label{eq:shrink}
    \boldsymbol{\hat{\beta}} = (\boldsymbol{X}^T\boldsymbol{X}+\lambda\boldsymbol{I})^{-1}\boldsymbol{X}^T\boldsymbol{y}
\end{equation}

In the proof we will use singular value decomposition. Using SVD of a symmetric matrix $\boldsymbol{X}^T\boldsymbol{X}$ and its properties, we can express $\boldsymbol{X}^T\boldsymbol{X}+\lambda\boldsymbol{I}$ as: 
\begin{equation}\label{eq:inv_proof}
\begin{split}
    \boldsymbol{X}^T\boldsymbol{X} + \lambda\boldsymbol{I}&=\boldsymbol{V}\boldsymbol{\Sigma}\boldsymbol{V}^T + \lambda\boldsymbol{I} = \boldsymbol{V}\boldsymbol{\Sigma}\boldsymbol{V}^T + \lambda\boldsymbol{V}\boldsymbol{V}^T\boldsymbol{I}\boldsymbol{V}\boldsymbol{V}^T \\ 
    = &\boldsymbol{V}(\boldsymbol{\Sigma} + \lambda\boldsymbol{V}^T\boldsymbol{I}\boldsymbol{V})\boldsymbol{V}^T = \boldsymbol{V}(\boldsymbol{\Sigma} + \lambda\boldsymbol{I})\boldsymbol{V}^T 
\end{split}
\end{equation}
Let us notice that $\boldsymbol{V}(\boldsymbol{\Sigma} + \lambda\boldsymbol{I})\boldsymbol{V}^T$ is invertible, because $\boldsymbol{V}$ is invertible (it is orthogonal) and $\boldsymbol{\Sigma} +\lambda\boldsymbol{I}$ is invertible as well (this is a diagonal matrix and we have certainty that it does not have zeros on its diagonal).

\newpage
\section{CCA computing algorithm based on SVD decomposition}
\label{app:cca}

The algorithm based on SVD is not the fastest way of computing CCA. However, this approach is clear and shows the idea which is behind CCA. Let $\boldsymbol{X}$ and $\boldsymbol{Y}$ be $(m \times n)$ and $(m \times k)$ matrices. The first step of the method is to find SVD decomposition of $\boldsymbol{X}$ and $\boldsymbol{Y}$:
\begin{equation}
    \boldsymbol{X}=\boldsymbol{U}_1\boldsymbol{S}_1\boldsymbol{V}^T_1, \quad \boldsymbol{X}=\boldsymbol{U}_2\boldsymbol{S}_2\boldsymbol{V}^T_2
\end{equation}
Next, a matrix $\boldsymbol{U}_1^T\boldsymbol{U}_2$ is formed and also decomposed by SVD:
\begin{equation}
    \boldsymbol{U}_1^T\boldsymbol{U}_2 = \boldsymbol{Q}\boldsymbol{\Sigma}{\boldsymbol{L}^T}
\end{equation}
Finally, matrices of coefficients of linear combinations for $\boldsymbol{X}$ and $\boldsymbol{Y}$ are computed \citep{William}:
\begin{equation}
    \boldsymbol{W}_x=\boldsymbol{V}_1\boldsymbol{S}_1^{-1}\boldsymbol{Q}, \quad  \boldsymbol{W}_y=\boldsymbol{V}_2\boldsymbol{S}_2^{-1}\boldsymbol{L}
\end{equation}


\chapter{Multi-label classification library in practice}
\section{Example of usage of Armadillo library}
\label{app:arma}
\lstinputlisting[numbers=left, language=C++, caption=armadillo.cpp]{listings/armadillo.cpp}

\newpage
\section{Example of data in ARFF format}
\label{app:arff}
\lstinputlisting[caption=iris.arff]{listings/iris.arff}
\lstinputlisting[caption=iris.xml, language=xml]{listings/iris.xml}

\newpage
\section{Example of learning flow}
\label{app:learning}
\lstinputlisting[numbers=left, language=C++, caption=learning.cpp]{listings/learning.cpp}

\newpage
\section{Example of evaluation flow}
\label{app:evaluating}
\lstinputlisting[numbers=left, language=C++, caption=evaluation.cpp]{listings/evaluation.cpp}

\end{appendices}


% Bibliography (books, articles) starts here.
\bibliographystyle{alpha}{\raggedright\sloppy\small\bibliography{bibliography}}

% Colophon is a place where you should let others know about copyrights etc.
\ppcolophon

\end{document}
