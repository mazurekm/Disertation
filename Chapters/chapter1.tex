
\chapter{Introduction}

\section{Motivation}

It is hard to disagree that we live in `big data' era. Data centers are processing a huge amount of information in order to find new trends, connections or even predict some phenomenons in every aspect of our life. Most of us has probably noticed suggestions, which appear while using social media or doing online shopping. It is sometimes amazing how accurate these suggestions are. In fact, applications learn our habits from data that we leave while surfing the Internet. The IT world is becoming closer and closer something, we can call true artificial intelligence. In my opinion the most suprising issue is that, there is no need to use sophisticated algorithms in order to achieve such promising results. It is caused by huge amount of data we can access and fast computing systems, which have become more significant than mathematical models.  

Besides predicting trends or making suggestions, we could also use AI algorithms is some specific classification tasks. Let us imagine, there is an application which might assign hashtags to our post on Twitter or Instagram after analysing the text first. Or an application, which could assign categories to our article on Wikipedia. This paper is devoted to algorithms, which could be used to implement the applications, that make this kind of tasks. 

\section{Problem discussion}

Machine learning is a branch of computer science which focuses on creating an automatic system, which can learn and improve itself from gathered experience (data). There are three main types of machine learning tasks:

\begin{itemize}
\item Supervised learning
\item Unsupervised learning
\item Reinforcement learning
\end{itemize}

In supervised learning the experience is represented by a set of an example input data with desired outputs. This set is provided by a "teacher" and the main goal is to find rules, which allow to predict output from given input. Reinforcement learning is a more complicated issue. In this case we usually consider a dynamical environment with various states and an agent which can perform one of available actions in order to achieve a goal. In contrast to supervised learning, there is no information about correct input/output pairs - there is no "teacher". As a result the environment must be explored by the agent in order to gain a knowledge. In other words instead of "teacher" there is a "nitpicker" which awards an agent for making a particular action. The last type of machine learning task - unsupervised learning aims at finding hidden structures in unlabeled data. In this case there is no any error or reward signal which could be used to evaluate a solution. In fact this is the most challenging aspect of machine learning. 

This study is devoted to multi-label classification which is one of basic problems related to supervised learning. 

The classification aims at matching specific category (also called class) to a new observation base on a traning set. The training set is a set of observations for whom the class, they belong to, is known. Let us notice that the classification can be considered as two separated problems: binary classification and multiclass classification. The difference between them is certainly connected with the number of classes. In the first case we have only two categories, whereas in the second one there are many of them. In other words, the mutlticlass classification is an extension of the binary classification for more than two categories, however there is always one target - the only one class is assigned to the object. If we increase the number of targets instead of the number of classed in binary classication, then we get multi-label classification problem. This time a sequence of labels is predicted for a specific instance. Let us notice that there is also a relation between multiclass and multi-label classification. We could treat each sequence of labels as an independent category. However, the number of classes grows exponentially with the increasing number of labels, what makes this approach useless. 

\section{Scope of thesis}

The scope of the thesis involves fast implementation of multi-label classification algorithms which base mainly on linear algebra axioms. Let us notice that matrix representation of training data is comfortable and a kind of intuitive. Moreover, linear algebra operations such as matrices multiplication or matrix invertion can be effectively performed in multi-thread environment.  As a result even for huge training sets, there might be a sensible time of building classifier achieved. 

The implementations were made in \textit{C++11} programming language. This decision was dictated by efficiency profit of using low-level programming language and a full control over compilation process. Unfortunately there are no  \textit{C++} libraries that contain parsers for a popular data format (i.e. \textit{ARFF}) or tools which measure quality of classification. In order to make experiments and to test the solutions, such additional tools had to be implemented as well. As a result these all elements were collected together and create a library which can be installed in an operating system.


