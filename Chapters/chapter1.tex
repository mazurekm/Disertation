
\chapter{Introduction}

\section{Motivation}

It is hard to disagree that we live in a 'big data' era. Data centers are processing a large quantity of information in order to find new trends, connections or even predict some phenomenons in every aspect of our life. Most of us has probably noticed suggestions which appear while using social media or doing online shopping. It is sometimes amazing how accurate these suggestions are. In fact, applications learn our habits from data that we leave while surfing the Internet. We are approaching something what we can call true \textit{Artificial Intelligence}. In my opinion the most suprising issue is a fact that there is no need to use sophisticated algorithms in order to achieve such promising results. It is caused by massive data we can access and fast computing systems which have become more significant than mathematical models. 

Besides predicting trends or making suggestions, we could also use \textit{AI} algorithms is specific classification tasks. Let us imagine that there is an application which might assign hashtags to our post on Twitter or Instagram after analysing the text. Or an application which could assign categories to our article on Wikipedia. This paper is devoted to algorithms which could be used to implement the applications that perform such classification tasks. 

\section{Problem discussion}

\textit{Machine Learning} is a branch of computer science which focuses on creating an automatic system which can learn and improve itself from a gathered experience (data). There are three main types of \textit{Machine Learning} tasks:

\begin{itemize}
\item supervised learning,
\item unsupervised learning,
\item reinforcement learning.
\end{itemize}

In the supervised learning the experience is represented by a set of exemplary input data with desired outputs. This set is provided by a 'teacher' and the main goal is to find rules which allow to predict an output from a given input. The reinforcement learning is a more complicated issue. In this case we usually consider a dynamical environment with various states and an agent which can perform one of available actions in order to achieve a goal. In contrast to the supervised learning, there is no any information about correct input-output pairs. As a result, an environment must be explored by an agent in order to gain a knowledge. In other words instead of a 'teacher' there is a 'nitpicker' which awards an agent for making a particular action. The last type of \textit{Machine Learning} tasks - the unsupervised learning aims at finding hidden structures in unlabeled data. In this case there are no error and reward signals which could be used to evaluate a solution. In fact, this is the most difficult and also the most challenging aspect of \textit{Machine Learning}. 

This study is devoted to the multi-label classification which is one of basic problems related to the supervised learning. 

The general classification problem aims at matching a specific category (also called class) to a new observation basing a traning set. A training set is a set of observations for whom a class, they belong to, is known. Let us notice that in the classification problem we can distinguish two separated problems: the binary classification and the multiclass classification. The difference between them is certainly connected with a number of classes. In the first case we have only two categories, whereas in the second one there are many of them. In other words, the mutlticlass classification is the extension of the binary classification for more than two categories. However, in both the cases there is always one target - an only one class is assigned to an object. If we increase a number of targets instead of a number of classes in the binary classication, then we get the multi-label classification problem. This time a sequence of labels is predicted for a specific instance. Let us notice that there is also a relation between the multiclass classification and the multi-label classification. We could treat each sequence of labels as an independent category. However, a number of classes grows exponentially with an increasing number of labels what makes this approach useless. 

\section{Scope of thesis}

The scope of the thesis involves a fast implementation of multi-label classification algorithms which base mainly on linear algebra axioms. Let us notice that a matrix representation of training data is comfortable and intuitive as well. Moreover, the linear algebra operations such as matrices multiplication or matrix invertion can be effectively performed in a multi-threaded environment.  As a result, even for huge training sets, a sensible time of building classifier might be achieved. 

The implementations was made in \textit{C++11} programming language. This decision was dictated by an efficiency profit of using low-level programming language and a full control over a compilation process. Unfortunately there are no \textit{C++} libraries that contain parsers for a popular data format (i.e. \textit{ARFF}) or tools which measure a quality of classification. In order to conduct experiments and to test solutions, such additional tools had to be implemented as well. As a result, these all elements were collected together and became a part of the library which can be installed in an operating system.
