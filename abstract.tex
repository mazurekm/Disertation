
\begin{abstract}
\thispagestyle{plain}

Multi-label classification is one of fundamental tasks in \textit{Machine Learning} and is used in real world applications which categorize texts or label images. This thesis presents algorithms which can be used to solve multi-label problems. The main concept, common for each of presented methods, is to transform feature and label spaces to new linear spaces, and then to use regression analysis in order to train a classifier. The main aim of the thesis is to check how such the algorithms deal with various data sets.

In the first chapter, the thesis puts across what is multi-label classification from \textit{Machine Learning} point of view. There is also an explenation why the methods based on preliminary linear transformations can be effective and why it contains adventages of the most popular approaches used in multi-label classification. The thesis then shows mathematic background which is behind algorithms and presents their pseudocodes. In a central section, there is a detailed description of implementation of these methods in \textit{C++} programming language. It is worth emphesising that the implementation can work in a multithreaded environment. In the fourth chapter, the thesis illustrates results of quality and efficiency experiments with various data sets.  

The results of the experiments show that the algorithms adapt well to multi-threaded environment. In fact, the efficiency profit is noticible. Unfortunately, linear transformations cost time and usually affect negatively the quality of prediction. However, there is a potential in such the approach and there is still a lot to investigate.

\end{abstract}
\cleardoublepage
